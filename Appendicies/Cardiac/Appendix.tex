\chapter{Cardiac Disease}
\label{App:Cardiac}

\section{Outline of Background Information and Motivation}
Below is an outline which provides a more complete background to the adhesion studies using bone marrow mesenchymal stem cells with respect to the motivation behind the potential therapy and a role for a new tool to study adhesion. This background was provided by Eric Schmuck, the primary collaborator on this project.
\begin{outline}
\1 Heart disease is the leading cause of death in United States \cite{Heron:2009kx}
\1Including health services, medications and lost productivity, the estimated cost of Heart disease in 2010 was \$316.4 Billion \cite{Lloyd-Jones:2010vn}.
\1 Coronary Artery disease (CAD), which is a narrowing of vessels due to a build up of plaque (cholesterol deposits), is the main cause of myocardial infarction (MI).
\1 MI is defined as myocardial cell death due to prolonged ischaemia \cite{Thygesen:2007ys}.
\1 Myocardial cell death is not immediate following ischaemia, but takes approximately 6 hrs before the evidence of cardiomyocyte death can m be observed \cite{Thygesen:2007ys,Blankesteijn:2001zr}.
\1The length of ischaemia will generally correlate to size of the infarct, with complete necrosis of all “at risk” myocardial cells requiring between 2-4 hrs of occlusion \cite{Thygesen:2007ys}. But, there are many factors that go into determing the size of the infarct including presence of collateral circulation to the ischaemic zone, degree and continuity of blockage, sensitivity of the myocytes to ischaemia and the individual demand for oxygen and nutrients  \cite{Thygesen:2007ys,Alpert:2000ly}.
\1 Size of the infarct is important for determining the degree of post infarction cardiac remodeling, function and risk of developing heart failure \cite{Forrester:1976bh,McKay:1986ve,Pfeffer:1979qf}.
\1 Apoptosis is usually responsible for early cardiomyocyte death (6-8hr).  Necrosis (stressed induced swelling and lysing) is usually a secondary phenomenon occurring 12h to 4 days after MI \cite{Blankesteijn:2001zr}
\1There are 3-4 distinct phases of cardiac healing \cite{Cleutjens:1999fk,Frangogiannis:2008uq}
\2 Phase 1: Myocyte cell death (0-48 hrs)
\2 Phase 2: Acute Inflammation
\3 This phase is categorized by activation of the complement system and release of cytokines (IL-6, IL-8).  Within 6 hrs of the infarct, neutrophils migrate to the infarcted area with numbers peaking between 24-48 post MI to remove dead myocytes.  Lymphocytes, plasma cells and macrophages also migrate to the area to remove dead myocytes.
\2 Phase 3: Granulation tissue (3 days - 4 wks post MI)
\3 This phase is characterized by the deposition of new extracellular matrix proteins, first into the border zone between infarcted and non-infarcted tissue and later in the central area of the infarct.  The tissue becomes dense with cells (myofibroblasts, inflammatory cells, and highly vascularized)
\2 Phase 4: Remodeling and Repair (3 wks - 1 yr)
\3 This phase is characterized by accelerated ECM turnover and increased deposition in response to variation in wall stress.  Unique to heart healing is the persistence of fibroblasts in the scar area for long periods of time.  Fibroblasts have been found in the scar 17 yrs after infarct \cite{Willems:1994fk}.  After a scar is generated in skin, fibroblasts undergo apoptosis and the scar becomes devoid of cells.
\1 Due to a large number of the myocytes dying from Necrosis (which is messier than apoptosis), large amounts of cellular material are released into the extracellular space, triggering the inflammatory response \cite{Frangogiannis:2008uq}. 
This becomes important when attempting to treat an infarct with cellular reagents (stem cells). Nucleic acids and proteins can stick to the supporting extracellular matrix leaving a “dirty infarct” and potentially reduce the ability of stem cells to adhere in the infarcted area.
\2 As time progresses the immune response removes dead cells and debris.  Myofibroblasts make the granulation tissue by depositing copious amount of extracellular matrix proteins into damaged area.  This acts as strong patch for the remaining myocytes to pull against and resists bursting.
Due to the weakened state of the heart (mostly the inability of the heart to pump out all of the blood), pressure overload is common, this leads to changes in the non-infarcted tissue
\2 myocyte hypertrophy (within days of infarct), myocytes may increase there volume by more than 112\% \cite{Cleutjens:1999fk,Olivetti:1994kx}.
Endothelial cells proliferate to compensate for the increase in myocyte size
Increase in interstitial collagens to compensate for the increase in pressure load leads to stiffening of the heart
\2 To date, clinical trials utilizing stem cells have yield conflicting data with the best results showing modest improvements in cardiac function and the worst, no improvement \cite{Assmus:2010qf,Beitnes:2009vn,Erbs:2007ve,Meyer:2006ly,Meyer:2009zr,Schachinger:2006bh,Wollert:2004ys}.
\1 One of the many hurdles facing stem cell therapy for cardiac regeneration is low retention/engraftment of delivered cells.  Generally, less than 10\% of the cells delivered are retained in the heart, with many trials show less than a 1\% retention rate \cite{Freyman:2006nx,Ly:2009cr,Menasche:2010dq,Terrovitis:2009oq,Hofmann:2005tw}. 
\1 Fibroblasts are found in every tissue of the body, but the cells residing in those tissues are unique to that tissue \cite{Chang:2002ij,Fries:1994tg,Lekic:1997hc,Souders:2009kl}.  The ECMs they deposit are also unique to the tissue \cite{Chang:2002ij,Fries:1994tg,Lekic:1997hc,Souders:2009kl}.  For this reason, we have derived a 3-dimensional cardiac fibroblasts (CFB) derived matrix.  The matrix can be prepared to mimic either a recent infarct, that is the ECM proteins are contaminated with large amounts of intracellular proteins and nucleic acids, or a infarct in the granulation tissue phase, where much of the cellular debris has been removed large amounts of deposited extracellular matrix can be found.  The main components of the secreted ECM are Fibronectin, Collagen 1 and Collagen 3, with other minor ECM proteins.  Additionally, the matrix serves a repository for many cytokines and biologically active signaling molecules such as VEGF,VWF, EGF, TGFb, FGF, ILGF, HGFs to name a few \cite{Franitza:2000fu,Hynes:2009bs,Iyer:2008fv,Vaday:2000kl,Vaday:2000dz}.  These biologically active molecules could be important in directing binding and growth of cells on the matrix.  Recent data (as in yesterday) showed that BMSC (P1) cultured on a CFB derived matrix had proliferation rates more than double that of BMSC cultured at the same density on plastic (n=6/group).
\1 Cardiac Fibroblast (CFB) Isolation \cite{Baharvand:2005mi,Dubey:1997qa}:
\2 Male Lewis Rats (240-300g) are sacrificed by CO2 asphyxiation, hearts rapidly excised and atria removed and placed into ice cold PBS with 1\% penicillin/Streptomycin.  Hearts are finely minced then placed into 10mLs digestion media (DMEM, 73U/mL Collagenase 2, 12μg/mL Pancreatin (4x)) and incubated at 37°C with agitation for 35 minutes.  The digest mixture is centrifuged at 1000xg for 20 minutes at 4°C.  Resulting cell pellet is suspended in 10mLs of fresh digestion media and incubated at 37°C with agitation for 30 minutes.  The resulting digest is sieved through a 70μm cell strainer and digest solution diluted with 10mLs of Culture media (DMEM, 10\% FBS, 1\% Penicillin/Streptomycin).  The cell suspension is then centrifuged at 1000xg for 20 minutes at 4°C.  The cell pellet is suspended in 16mLs culture media and plated into 2 T75 culture flasks.  The cells are allowed to attach under standard culture conditions (37°C, 5\% CO2, 100\% humidity) for 2 hrs, then non-adherent cells removed by washing with PBS and culture media replaced. Primary CFB cultures are usually confluent in 4-7 days.
\1 Bone marrow mesenchymal stem cell isolation \cite{Baharvand:2005mi,Tropel:2004pi}:
\2 Male Lewis Rats (240-300g) are sacrificed by CO2 asphyxiation. Femurs and Tibias are bilaterally excised and soft tissue removed.  The bones are placed in ice cold PBS with 1\% penicillin/Streptomycin.  Ends of the bones are then removed and an 18 gauge needle and syringe used to flush the shafts of the bones with Culture media (DMEM, 10\% FBS, 1\% Penicillin/Streptomycin).  The resulting bone marrow is further dispersed by passage through a 21 gauge needle.  Cell suspension is then centrifuged at 1000xg for 10 minutes at 4°C and plated into a 100mm culture dish.  The cells are allowed to attach under standard culture conditions (37°C, 5\% CO2, 100\% humidity) for 24 hrs, then non-adherent cells removed by washing with PBS and culture media replaced.
\1 Components of the ECM are highly conserved among species.  Because of this they are well tolerated when grafted across species and do not tend to elicit an immune reaction \cite{Bernard:1983lh,Bernard:1983fu,Constantinou:1991ye,Exposito:1992qo,Gilbert:2006ff}.

\end{outline}

\section{Mass Spectrometry Data}
The following tables are preliminary 2-D mass spectrometry (MS) results provided by Eric Schmuck.  The technique uses a strong cation exchange column to elude off 7 fractions, which are each run through the MS column separately. Separating peptides based on charge first gives better resolution, however these results are only preliminary. The matches provided by the computer need to be verified. The computer program estimates there may be a 50\% false discovery rate based on the number of reversed proteins found.

\begin{table}[!ht]
\centering
\caption{\textbf{Mass spectrometry data table}.}
\includegraphics[width=5in]{MassSpecPage1.pdf}\\
\includegraphics[width=5in]{MassSpecPage2.pdf}\\
\end{table}

\begin{table}[!ht]
\centering
\caption{\textbf{Mass spectrometry summary table}.}
\includegraphics[width=5in]{MassSpecSummary.pdf}\\
\end{table}




