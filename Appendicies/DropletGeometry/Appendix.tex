\chapter{Droplet Geometry}
\label{App:DropletGeometry}

There are many different ways to formulate the geometry of passive pumping between an input drop and an output drop. The aim of this appendix is to describe the droplet geometry in as many ways as possible so that it does not need to be significantly re-derived in the future. The following variables are used.

\begin{table}[htdp]
\caption{\textbf{Droplet geometry variable names and definitions}. The subscript $i$ denotes the number of the droplet/port/wetted area of interest.}
\centering
\begin{tabular}{ll}\toprule
Variable&Description\cr
\midrule
$a_{i}$&Radius of the port/wetted area\cr
$R_{i}$&Radius of curvature of the droplet\cr
$\R_{i}$&$R_{i}/a_{i}$ (dimensionless)\cr
$H_{i}$&Height of the droplet\cr
$\h_{i}$&$H_{i}/a_{i}$ (dimensionless)\cr
$C_{i}$&Vertical position of the center of the radius of curvature of the droplet\cr
$\C_{i}$&$C_{i}/a_{i}$ (dimensionless)\cr
$V_{i,hemi}$&$\frac{2}{3}\pi a_{i}^{3}$\cr
$V_{i}$&Volume of the droplet\cr
$\V_{i}$&$V_{i}/V_{i,hemi}$ (dimensionless)\cr
$\theta_{i}$&Contact angle of the droplet with the device\cr

\bottomrule
\end{tabular}
\end{table}%

First we define relationships between the various variables of interest.

\section{\texorpdfstring{Radius of Curvature, $R_{i}$ and $\R_{i}$}{Radius of Curvature, Ri and Ri}}%%%%%%%%%%%%%%

\begin{equation}
%\centering
\R_{i}=\frac{R_{i}}{a_{i}}
\end{equation}

\begin{equation}
%\centering
\R_{i}=\sin^{-1}(\theta_{i})
\end{equation}

\begin{equation}
%\centering
\R_{i}=\frac{\h_{i}^{2}+1}{2\h_{i}}
\end{equation}

\section{\texorpdfstring{Height, $H_{i}$ and $\h_{i}$}{Height, Hi and Hi}}%%%%%%%%%%%%%%%%%%%%

\begin{equation}
%\centering
\h_{i}=\frac{H_{i}}{a_{i}}
\end{equation}

\begin{equation}
%\centering
\h_{i}=\frac{1-cos(\theta_{i})}{sin(\theta_{i})}=tan\left(\frac{\theta_{i}}{2}\right)
\end{equation}

\begin{equation}
%\centering
\footnote{always produces result between 0 and $a$}\:
\h_{i}=\R_{i}\left(1-\sqrt{1-\frac{1}{\R_{i}^{2}}}\right)
\end{equation}

\begin{equation}
%\centering
\h_{i}=\left(\frac{1-\left(-2\V_{i}+\sqrt{1+4\V_{i}^{2}}\right)^{2/3}}{\left(-2\V_{i}+\sqrt{1+4\V_{i}^{2}}\right)^{1/3}}\right)
\end{equation}

\section{\texorpdfstring{Contact Angle, $\theta_{i}$}{Contact Angle, theta}}%%%%%%%%%%%%%%%%%%%%

\begin{equation}
%\centering
\footnote{always produces result between 0\Deg and 90\Deg}\:
\theta_{i}=\arcsin\left(\frac{1}{\R_{i}}\right)
\end{equation}

\begin{equation}
%\centering
\theta_{i}=2\arctan\left(\h_{i}\right)
\end{equation}

\begin{equation}
%\centering
\theta_{i}=2\arctan\left(\frac{1-\left(-2\V_{i}+\sqrt{1+4\V_{i}^{2}}\right)^{2/3}}{\left(-2\V_{i}+\sqrt{1+4\V_{i}^{2}}\right)^{1/3}}\right)
\end{equation}

\section{\texorpdfstring{Volume, $V_{i}$ and $\V_{i}$}{Volume, Vi and Vi}}%%%%%%%%%%%%%%%%%%%

\begin{equation}
%\centering
\V_{i}=\frac{V_{i}}{V_{i,hemi}}
\end{equation}

\begin{equation}
%\centering
\V_{i}=\frac{1}{4}\h_{i}(3+\h_{i}^2)
\end{equation}

\footnote{always results in a normalized volume between 0 and 1 (\ie , $0\le V\le V_{hemi}$ or $0$\Deg$\le\theta\le90$\Deg)\label{fn:Vol}}
\begin{equation}
%\centering
\V_{i}=\frac{1}{4}\R_{i}\left(1-\sqrt{1-\frac{1}{\R_{i}^{2}}}\right)\left(3+\R_{i}^{2}\left(1-\sqrt{1-\frac{1}{\R_{i}^{2}}}\right)^{2}\right)
\label{equ:VofR1}
\end{equation}

\footnote{always results in a normalized volume $> 1$ (\ie , $V_{i,hemi}\le V_{i}\le\infty$ or $90$\Deg$\le\theta_{i}\le180$\Deg)}
\begin{equation}
%\centering
\V_{i}=2\R_{i}^3 - \frac{1}{4}\R_{i}\left(1-\sqrt{1-\frac{1}{\R_{i}^{2}}}\right)\left(3+\R_{i}^{2}\left(1-\sqrt{1-\frac{1}{\R_{i}^{2}}}\right)^{2}\right)
\label{equ:VofR2}
\end{equation}

$^{\ref{fn:Vol}}$
\begin{equation}
%\centering
\V_{i}=\frac{V_{i}}{V_{i,hemi}}=\frac{1}{4}\tan\left(\frac{\arcsin\left(\frac{a_{i}}{R_{i}}\right)}{2}\right)\left(3+{\tan\left(\frac{\arcsin\left(\frac{a_{i}}{R_{i}}\right)}{2}\right)}^{2}\right)
\label{equ:VofR3}
\end{equation}

$^{\ref{fn:Vol}}$
\begin{equation}
%\centering
\V_{i}=\frac{1}{4}\tan\left(\frac{\theta_{i}}{2}\right)\left(3+{\tan\left(\frac{\theta_{i}}{2}\right)}^{2}\right)
\end{equation}

\begin{figure}[!ht]
\centering
\includegraphics[width=3.2in]{Theta.pdf}
\caption{\textbf{Dimensionless variable relationships in passive pumping}. Plot of $a/R$ and contact angle, $\theta$, with respect to normalized droplet volume, $\V$ (or $V_{i}/V_{i,hemi}$). Theta is measured at the contact point between the spherical drop and the horizontal surface extending toward the center of the port.}
\end{figure}

\section{\texorpdfstring{Location of Center of Radius of Curvature, $C_{i}$ and $\C_{i}$}{Location of Center of Radius of Curvature, Ci and Ci}}

\begin{equation}
%\centering
\C_{i}=\frac{C_{i}}{a_{i}}
\end{equation}

\begin{equation}
%\centering
\C_{i}=\h_{i}-\R_{i}
\end{equation}

\begin{equation}
%\centering
\C_{i}=-\R_{i}\sqrt{1-\frac{1}{\R_{i}^{2}}}
\end{equation}

\begin{equation}
%\centering
\C_{i}=-\R_{i}\cos\left(\theta_{i}\right)
\end{equation}


