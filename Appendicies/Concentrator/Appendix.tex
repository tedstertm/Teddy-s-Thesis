\chapter{Concentrator: Additional Information}
\label{App:Concentrator}

\section{Estimation of Cell Number Limit}

A lower limit of ~50,000 cells is calculated based on a seeding density of 250 cells/mm$^{2}$ for culture in a microchannel with a height of 250 \textmu m (\ie , a density of 1000 cells/\textmu L). It is also assumed that the original sample is centrifuged to a minimum volume of 50 \textmu L to avoid aspiration of any concentrated cells. The limit is specific to the application.

\section{Initial Design Approach for Concentrator Design}

The dimensions of the initial device design were chosen in order to produce a shear stress of roughly 0.1 dynes/cm$^{2}$ within the collection region and to give the cells a chance to settle to the substrate after entry into the collection region (i.e., t$_{set}/t_{res}$ $\approx$ 1). Transport channel dimensions were then varied to produce shear rates and residence times above and below these thresholds to explore the appropriate conditions for proper cell retention. Channel resistances were estimated using the Washburn Law and applied to an electrical circuit analog for fluid flow. Passive pumping pressures were estimated using Eq \ref{equ:laplace}.

Design II differs from Design I primarily in the dimensions of the collection region. The width was increased while the height was reduced based on the dimensional analysis described in the main text. Channel height does not affect t$_{set}/t_{res}$ but can affect shear. When channel width increases, t$_{set}/t_{res}$ decreases, thereby increasing the chance for cell capture. Thus Design II was viewed as an improvement to Design I.

\section{Cell Loss Modeling}

COMSOL Multiphyisics v3.4 (Burlington, MA) was used to obtain steady state solutions for fluid flow through Deisgn I and II. First a simulation of the whole device (using symmetry) was used to determine that the pressure drops across the transport channels (from center input to the outer ring) were within 0.55\% of one another (see Fig \ref{fig:evenPressure}). Because flow through each transport is nearly identical, the 3D model was simplified to include only one of the repeating sections of the collection region. This repeating section was then divided in half along its line of symmetry. This allowed a finer meshing of the region for more detailed analysis.

\begin{figure}[!b]
\centering
\includegraphics[width=3in]{EvenPressure_Composite.pdf}
\caption{\textbf{Plot of normalized pressure along device axis of symmetry}. The pressure at the input is defined as 1 whereas the pressure nearest the output port is defined as zero. The normalized pressure farthest from the output, at the intersection of the transport channel and outer ring, is then 0.0055.}
\label{fig:evenPressure}
\end{figure}

A threshold of 0.1 dynes/cm$^{2}$ was used to help determine cell loss (see Fig \ref{fig:shearStress}). If streamline calculations suggest that a cell settles to the surface before reaching the collection region exit, then the shear stress is calculated at that location. If the shear stress at that location is below the threshold of 0.1 dynes/cm$^{2}$, then the cells that follow that streamline are assumed to be captured. If the shear stress is too high, it is assumed the cells are lost. Only one half of the region depicted was simulated using a symmetry boundary condition along the center line. The results are reflected to better indicate the shape of the repeating portion of the collection region.

\begin{figure}[!b]
\centering
\includegraphics[width=3in]{ShearStress.pdf}
\caption{\textbf{Concentrator shear stress near the substrate}. Shear stress one cell radius (6.25 \textmu m) above the substrate for a total device flow rate of 5 \textmu L/min. The color legend indicates shear stresses between 0 and 0.1 dynes/cm$^{2}$. Simulations assume that cells are not collected in regions with a shear stress above 0.1 dynes/cm$^{2}$ (white areas near inlet and outlet). Flow velocities in the x-direction half way through the collection region are shown as a slice heat map.}
\label{fig:shearStress}
\end{figure}

Streamlines were modeled by assuming that the cell settles at a rate of 2.7 \textmu m/s, a value obtained using the Stokes drag of the particle. Streamlines were calculated using COMSOL starting at the entrance into the collection region (see Fig \ref{fig:streamlines}). The starting points form a grid in the entrance. The size of the cell prohibits the center of the cells from following streamlines that are less than a radius from the transport channel wall. Thus the grid of streamline start-points begin one radius in from the transport channel walls. The streamlines are followed until the ``cell'' exits the modeled space (\ie , the substrate or collection region exit).

\begin{figure}[!t]
\centering
\includegraphics[width=3in]{StreamlinePlot.pdf}
\caption{\textbf{Particle streamlines within the concentrator}. Example plot of streamlines used to calculate percent loss for a given flow condition and device design. Streamlines begin from a grid of locations at the entrance to the collection region. The streamlines are followed until they exit the collection region or intersect with the substrate of the device. These streamlines are calculated assuming the cell settles at a velocity of 2.7 \textmu m/s and used in further calculations to determine if the cell is captured or lost. A high density of streamlines is used to increase the precision of the simulated loss measurements.}
\label{fig:streamlines}
\end{figure}

By numerically integrating the inverse of the particle velocity along the length of a streamline, one can obtain the time it takes for the particle to reach a location along that streamline. The amount of fluid that has flowed along the streamline is equal to the flow rate at the starting point times the time that has been allowed to pass. The flow rate at the starting point of the streamline is dictated by the fluid velocity and the grid spacing for the streamlines. The information regarding the time for the particles to reach a location along the streamline can be converted to volume information using this flow rate. In other words, the data is transformed such that it represents how much volume must flow into the device to move the particles to a certain position along a streamline. The volume-position data for each streamline is then used to determine how many cells would be lost for a given streamline, cell suspension density, and volume that has been pumped at a predetermined average flow rate. This method simplifies pumping as a square wave instead of a natural passive pumping flow profile. The average flow rate used to calculate cell loss is determined by taking the volume-averaged flow rate of a passive pumping profile that matches the given experimental condition. The method described here does take into account fixed-volume effects as described in the main article. If not enough fluid has been put through the device to cause any cells to leave the collection region on a streamline, all the cells on that streamline are considered captured.

\section{Experimental Calculation of \texorpdfstring{{\boldmath$t_{set}/t_{res}$}}{tset tres}}

Passive pumping is a dynamic phenomena where pressure changes with the geometry of the droplet that drives fluid flow. In order to estimate a value for $Q$ from this dynamic process for use in Eq \ref{equ:ratio}, an `average' value must be chosen.  A volume-averaged flow rate is chosen instead of a time-averaged flow rate as we are interested in the average velocity that a particular unit volume sees within the collection region. The volume-averaged flow rate can be calculated from the volume-averaged pressure, $P$, and resistance, $Z$, given that $P=QZ$ for low Reynolds number flow. The volume-averaged pressure can be determined using the relationship between droplet geometry and pressure given by the Young-Laplace equation (Eq \ref{equ:laplace}). The resistance of the device was determined using experimental measurements of passive pumping times and droplet geometries, as measured using a goniometer. The data was matched with an analytical solution for passive pumping\cite{Berthier:2007mi} to determine the device resistance, $Z$, needed to produce the appropriate pumping time given the initial droplet geometry. This resistance, volume-averaged pressure, and device geometry ($A$) are then used to calculate an appropriate flow velocity, $v_{f}$, for estimating $t_{set}/t_{res}$. The cross sectional area, A, is calculated at the average radius of the device collection region, $r_{d}$ = 3.75 mm. 

Calculations of flow rate in Design II are summarized in Fig \ref{fig:pp}. Instead of plotting pressure \vs\ volume, device geometry and resistance are taken into account to plot flow rate \vs\ volume. Results for a 6 \textmu L and 15 \textmu L drop are shown as these are the two different sizes of drops used experimentally. Simulated cell trajectories are depicted in a cross-section view of the collection region and show how higher flow rates cause more cell streamlines to flow out of the collection region instead of reaching the substrate. The trajectories are simulated using steady-state conditions and are only shown for those along the centerline of the transport channels.
