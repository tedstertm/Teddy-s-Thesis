\chapter{Oscillatory Flow}
\label{App:Oscillator}

Below is a table with the constants and variables used throughout this appendix with their definitions.

\begin{table}[!ht]
\caption{\textbf{Variables and constants of fluid flow and their definitions}.}
\centering
\begin{tabular}{cp{10.5cm}}\toprule
Variable/Constant & Description \cr \midrule
$p$ & pressure \cr
$\delta$ & constant pressure gradient amplitude \cr
$\delta_{o}$ & oscillating pressure gradient amplitude \cr
$\rho$ & fluid density \cr
$\mu$ & dynamic viscosity of the newtonian fluid \cr 
$\nu$ & kinematic viscosity, $=\mu/\rho$ \cr
$f$ & frequency of the oscillatory pressure gradient \cr
$\omega$ & radial frequency, $=2\pi f$ \cr 
$h$ & the full height of the channel \cr
$\lambda$ & penetration depth, $=\sqrt{2\nu/\omega}$ \cr 
$u$ & fluid velocity \cr
$\bar u$ & average fluid velocity \cr
$u_{max}$ & maximum fluid velocity \cr
$\tau_{wall}$ & fluid shear stress at the top and bottom surface of the channel \cr
$\tau_{wall, o}$ & fluid shear stress at the top and bottom surface of the channel due to the oscillatory component of flow\cr
$\gamma_{wall}$ & fluid shear rate at the top and bottom surface of the channel \cr
\bottomrule
\end{tabular}
\label{tab:}
\end{table}

We first present the solution to steady laminar flow between parallel plates as it will be references later in the context of oscillatory flow.

\section{Steady Laminar Flow Between Parallel Plates}
With the origin positioned half way between the plates (\ie , $y=0$) and the full height of the channel defined as $h$, the equation for $u$, the velocity in the x-direction is as follows.

\begin{equation}
\centering
u= \frac{1}{2\mu}\left[\frac{d}{dx}(p+\rho g z)\right]\left(\left(y-\frac{h}{2}\right)^{2}-\frac{h^{2}}{4}\right)
\label{equ:u}
\end{equation}
The flow rate through the cross section, where $b$ is the width that is $\gg h$ is as follows.
\begin{equation}
\centering
Q=-\frac{bh^{3}}{12\mu}\left[\frac{d}{dx}(p+\rho g z)\right]
\label{equ:Q}
\end{equation}
The average velocity of fluid between the plates is
\begin{equation}
\centering
\bar u = -\frac{h^{2}}{12\mu}\left[\frac{d}{dx}(p+\rho g z)\right] = \frac{Q}{bh} = \frac{2}{3}u_{max}.
\label{equ:ubar}
\end{equation}
The wall shear stress is given by the following.
\begin{equation}
\centering
\tau_{wall} = -\frac{h}{2} \left[\frac{d}{dx}(p+\rho g z)\right] = \frac{6\mu Q}{bh^{2}} = 2\mu \gamma_{wall}
\label{equ:tauWall}
\end{equation}

\section{Oscillatory, Non-Turbulent Flow Between Parallel Plates}

The mathematical analysis of oscillatory flow in a microchannel that can be approximated as a parallel plate flow chamber is presented here along with analysis of how much the surface tension at the ports of a tubeless microchannel influences flow. The following closed form solutions for oscillatory flow are reproduced from Bacabac \etal, 2005 \cite{Bacabac:2005ax} who compiled the information from Landau \etal, 1959 \cite{Landau:1959vh}. 

The equation for the velocity $u$ at vertical position $y$ in the channel at time $t$ where $y=0$ is half-way between the plates and h is the distance between the plates is given by Eq \ref{equ:uofyt} with subelements given by the equations that follow.

\begin{equation}
u(y,t)=C_{1}(y)+C_{2}(y)cos(\omega t)+C_{3}(y)sin(\omega t)
\label{equ:uofyt}
\end{equation}

\begin{equation}
C_{1}(y)=-\frac{\delta}{2\mu}\left(y^{2}-\frac{h^{2}}{4}\right)
\label{equ:C1}
\end{equation}

\begin{equation}
C_{2}(y)=\frac{\delta_{0}}{\rho\omega}\left[\frac{CC\left(\frac{-y+\frac{h}{2}}{\lambda},\frac{y+\frac{h}{2}}{\lambda}\right)+CC\left(\frac{y+\frac{h}{2}}{\lambda},\frac{-y+\frac{h}{2}}{\lambda}\right)}{cc_{+}\left(\frac{h}{\lambda}\right)}-1\right]
\label{equ:C2}
\end{equation}

\begin{equation}
\footnote{I believe that there is a typo in this equation of Bacabac \etal . The discrepancy is in the signs of the arguments of the second SS function. Otherwise, the results do not match the results in the paper or those obtained using COMSOL simulation as a secondary check.}C_{3}(y)=\frac{\delta_{0}}{\rho\omega}\left[\frac{SS\left(\frac{-y+\frac{h}{2}}{\lambda},\frac{y+\frac{h}{2}}{\lambda}\right)+SS\left(\frac{y+\frac{h}{2}}{\lambda},\frac{-y+\frac{h}{2}}{\lambda}\right)}{cc_{+}\left(\frac{h}{\lambda}\right)}\right] 
\label{equ:C3}
\end{equation}

\begin{equation}
\lambda = \sqrt{\frac{2\nu}{\omega}}
\label{equ:lambda}
\end{equation}


\begin{equation}
\begin{array}{r@{=}l}
SS(x_{1},x_{2})&\sin(x_{1})\sinh(x_{2}) \cr
CC(x_{1},x_{2})&\cos(x_{1})\cosh(x_{2}) \cr
ss_{\pm}(x)&\sin(x)\pm\sinh(x) \cr
cc_{\pm}(x)&\cos(x)\pm\cosh(x) \cr
\end{array}
\label{equ:C4}
\end{equation}

From these solutions, Bacabac \etal\ defined correction factors for the amplitude of the fluid flow rate and wall shear stress amplitude by normalizing results by the value predicted for static flow. Thus, when the correction factors are significantly different than 1, static solutions to the flow are no longer good approximations and dynamic effects are becoming significant. $T$ is the wall shear stress correction factor (when the flow rate is known) while $K$ is the correction factor for the amplitude of the flow rate, $q$ when the pressure gradient is known. Thus, if one is predicting shear when the pressure gradient is known, one must correct the flow rate using $K$ and plug the adjusted flow rate into the equation for shear which is then adjusted by $T$.

I was unable to successfully reproduce results given the Bacabac version of the equation for $K$ but was able to implement $T$ and the product $TK$, from which $K$ can be obtained through division. The following equations are used to predict flow and wall shear stress due to the oscillatory component of flow.

\begin{equation}
T(h/\lambda) = \frac{1}{3\sqrt{2}}  \frac{\frac{h}{\lambda}}{cc_{+}(\frac{h}{\lambda})}   \sqrt{\frac{(\sin(h/\lambda))^{2}+(\sinh(h/\lambda))^{2}}{1-\frac{2(\frac{\lambda}{h})ss_{+}(h/\lambda)}{cc_{+}(h/\lambda)}-\frac{2(\frac{\lambda}{h})^{2}cc_{-}(h/\lambda)}{cc_{+}(h/\lambda)}}}
\label{equ:T}
\end{equation}

\begin{equation}
T(h/\lambda)K(h/\lambda) = \frac{\sqrt{\left(ss_{-}(h/\lambda)\right)^{2}+\left(ss_{+}(h/\lambda)\right)^{2}}}{\frac{h}{\lambda}\;cc_{+}(h/\lambda)}
\label{equ:TK}
\end{equation}

\begin{equation}
q_{o}=\frac{bh^{3}\gamma_{o}}{12\mu}K(h/\lambda)
\label{equ:q}
\end{equation}

\begin{equation}
\tau_{wall, o}=\frac{6\mu q_{o}}{bh^{2}}T(h/\lambda)=\frac{h\gamma_{o}}{2}T(h/\lambda)K(h/\lambda)
\label{equ:tau}
\end{equation}

Bacabac \etal\ point out in their paper that $T$ does not vary significantly for $h/\lambda<2$ while $K$ does not vary significantly for $h/\lambda<1$. The thresholds are different because inertial effects on flow rate can be felt before the parabolic flow profile is significantly altered, at which point the shear stress for a given flow rate is altered. Therefore, the threshold of $h/\lambda<2$ is to be used as a predictor of quasi-parabolic flow whereas the threshold of  $h/\lambda<1$ should be used to determine when steady flow can be assumed to provide accurate predictions for a given instant in time.

Using the threshold of $h/\lambda<2$, one can predict the critical frequency, $f_{c}$, of a system using Eq \ref{equ:fcrit};

\begin{equation}
f_{c} = \frac{4\mu}{\rho\pi h^{2}}
\label{equ:fcrit}
\end{equation}

It should also be noted that Bacabac also cites Loudon \etal\ who empirically related the Womersley number (a number that characterized the balance of viscous and inertial forces in oscillatory flow) and the Reynolds number, $Re$ \cite{Loudon:1998sf}. Loudon \etal\ found that the Reynolds number was approximately 150-250 times the Womersley number in the blood vessels of dogs. Using this relation and assuming that the transition to turbulence occurs near $Re=2000$, we can predict that turbulence in the parallel plate flow chamber would occur at around $h/\lambda=8/\sqrt{2}\approx5.7$. Thus, we can also characterize the turbulence of the system using the same dimensionless quantity $h/\lambda$.


%(1/(sqrt(2)*3)) * (r/fun('cc+',r,0)) * (sqrt(((sin(r))^2+(sinh(r))^2)/(1-(2*(1/r)*fun('ss+',r,0))/(fun('cc+',r,0))-(2*(1/r)^2*fun('cc-',r,0))/(fun('cc+',r,0)))))

\section{Steady Flow in a Rectangular Duct}

A relatively simple expression that agrees with resistance in rectangular ducts is given by the Washburn law, taken from and reproduced here in Eq \ref{equ:Washburn} from \cite{Shah:1978fb}. Eq \ref{equ:Washburn} simplifies to the case of flow between parallel plates when the aspect ratio ($\lambda=$width/height$ = w/h$) of the rectangular duct is $>4.45$. The variable L is the length of the duct and $\mu$ is the dynamic viscosity of the fluid.

\begin{equation}
\centering
K=\frac{8\,\mu\,L\,g(\lambda)}{h^{3}\,w}
\label{equ:Washburn}
\end{equation}

\begin{displaymath}
\centering
g=\left\{\begin{array}{cl}\frac{3}{2}&\textrm{if }\lambda>4.45\cr\frac{(1+\lambda)^{2}}{\lambda^{2}}&\textrm{if }\lambda<4.45\cr\end{array}\right.
\end{displaymath}

The following correction factor for shear stress in a rectangular duct is taken from \cite{NATARAJA.NM:1973fk}.
\begin{equation}
\centering
\tau_{w}= \frac{6 \mu Q}{wh^{2}}\frac{m+1}{m} \textrm{, for $h/w < 0.33$}
\end{equation}

\begin{equation}
\centering
m = 1.7 + 0.5 \left(\frac{h}{w}\right)^{-1.4}
\end{equation}

\section{Shear Stress from Amplitude}

You don't need to use a correction factor to find shear stress for the center portion of a rectangular duct if you measure amplitude of a particle in motion. This is because in the center portion, the amplitude of the particle gives an accurate indication of the maximum velocity for the nearly parabolic flow in that region. If you measure the amplitude, $A$, where particle position is $Asin(2\pi ft)$ and know the frequency, $f$, you can use Eq \ref{equ:amp} to get the shear stress, $\tau_{w}$.

\begin{equation}
\centering
x=A \sin (2\pi f t)
\label{equ:amp}
\end{equation}

\begin{equation}
\centering
\frac{dx}{dt} = u = 2 \pi A f \cos (2\pi f t)
\label{equ:dxdt}
\end{equation}

\begin{equation}
\centering
u_{max} = 2 \pi A f
\label{equ:umax}
\end{equation}

Given,

\begin{equation}
\centering
\bar u = \frac{Q}{bh} = \frac{2}{3}u_{max}
\label{equ:ubar2}
\end{equation}

and

\begin{equation}
\centering
\tau_{wall} = \frac{6\mu Q}{bh^{2}}
\label{equ:tauWall2}
\end{equation}


then,

\begin{equation}
\centering
\tau_{wall} = \frac{Q}{bh}\frac{6\mu}{h} = \frac{2}{3}u_{max}\frac{6\mu}{h} = \frac{2}{3}2 \pi A f \frac{6\mu}{h} = \frac{8 \pi A f \mu}{h}
\label{equ:tauWall3}
\end{equation}

\begin{equation}
\centering
\tau_{w}= \frac{8 \pi A f \mu}{h}
\label{equ:tauWall4}
\end{equation}

In order to find the amplitude desired for a given geometry and shear stress, the equation is rearranged as follows.

\begin{equation}
\centering
A = \frac{h \tau_{w}}{8 \pi f \mu}
\end{equation}

For a channel height of 140 $\times$ $10^{-6}$ [m] and a viscosity of 0.00078 [kg/ms] the equation simplifies to the following.

\begin{equation}
\centering
A = 0.00714 \frac{\tau_{w}}{f} \textrm{ [Pa]}
\end{equation}

\section{Units and Conversions}

Water is 1 cP or 0.01 P.
\begin{equation}
\centering
1 \textrm{ Poise} = 1 \textrm{ g/cm\,s} = 0.1\textrm{ kg/m\,s}
\end{equation}

\begin{equation}
\centering
1 \textrm{ Pascal or kg\,m/s}^{2} = 10 \;\mathrm{ dynes/cm}^{2} \textrm{ or g/cm\,s}^{2}
\end{equation}

\begin{equation}
\centering
1 \;\mathrm{ Newton} = 100\,000 \;\mathrm{ dynes}
\end{equation}

\begin{equation}
\centering
1 \;\mathrm{ kg/m\,s} = 10 \textrm{ P or g/cm\,s}
\end{equation}

Therefore multiply $\partial v_{x}/\partial y$ of units [1/s] by $\mu$, 0.00078 [kg/ms] for DMEM with supplements, to get shear stress in Pa and then multiply the shear stress that is in [Pa] by 10 to get [dynes/cm$^{2}$].

\section{Matlab Code}

\begin{matlab}
\caption{Matlab code that defines the $C_{1}$, $C_{2}$, and $C_{3}$ of Eq \ref{equ:u}.}
\lstinputlisting{/Users/warrick/Documents/MMB_Lab/Papers_and_Chapters/Thesis_Materials/Prelim/appOscillatoryFlow/Materials/C.m}
\end{matlab}

\begin{matlab}
\caption{Matlab code that defines the functions of Eq \ref{equ:C4}.}
\lstinputlisting{/Users/warrick/Documents/MMB_Lab/Papers_and_Chapters/Thesis_Materials/Prelim/appOscillatoryFlow/Materials/fun.m}
\end{matlab}

\begin{matlab}
\caption{Matlab code that defines the shear corrections factor $T$ of Eq \ref{equ:T}.}
\lstinputlisting{/Users/warrick/Documents/MMB_Lab/Papers_and_Chapters/Thesis_Materials/Prelim/appOscillatoryFlow/Materials/T.m}
\end{matlab}

\begin{matlab}
\caption{Matlab code that defines the product of the shear and flow correction factors $T$ and $K$. The product is shown in Eq \ref{equ:TK}.}
\lstinputlisting{/Users/warrick/Documents/MMB_Lab/Papers_and_Chapters/Thesis_Materials/Prelim/appOscillatoryFlow/Materials/C.m}
\end{matlab}


%Oscillatory flow can be created in an tubeless microchannel via shaking. Shaking a microchannel induces an acceleration of the microchannel and thus a pressure gradient within the channel. Due to the typically viscous dominated character of fluid flow in microchannels, the flow is typically laminar and parabolic; however, frequencies and amplitudes of oscillation can be such that the fluid flow is dominated by inertia. Further, there is another category of oscillatory flow in a tubeless channel, which is dominated by surface tension considerations. The particulars of surface tension based pressures and droplet geometry are contained in App \ref{appSurfaceTension}. Equations of flow in a microchannel that is viscous dominated are contained here and are reproduced from Bacabac \etal, 2005 \cite{Bacabac:2005ax}.

%We hope to analyze the importance of surface tension when considering a microchannel that is oscillated. For this we will consider simple oscillatory flow in a rigid tube. First, we will model the system for low frequencies (i.e. the system is dominated by viscous forces and there is a 180$^{\circ}$ phase shift between pressure gradient and and parabolic fluid flow). This situation can be described with the following equations.
%\begin{equation}
%Q = \left(-\frac{\partial P}{\partial x}\right)\frac{\pi R^{4}}{8 \mu}
%\end{equation}
%\begin{equation}
%V_{max} = \left(\frac{\partial P}{\partial x}\right)\frac{R^{2}}{4 \mu}
%\end{equation}
%\begin{equation}
%V = V_{max}\left(1-\left(\frac{r}{R}\right)^{2}\right)
%\end{equation}
%\begin{equation}
%\tau =\left( -\frac{\partial P}{\partial x}_{max}\right)\frac{R}{2}
%\end{equation}

%For small oscillations in volume, the pressure gradient from surface tension can be approximated with a 'spring constant' that relates pressure to volumetric displacement. The maximum possible value of the spring constant, $\partial P/\partial V$, is shown in Eq \ref{equ:k}. 
%\begin{equation}
%\frac{\partial P}{\partial V} = \frac{8 \gamma}{\pi a^{3}}
%\label{equ:k}
%\end{equation}
%Because the inertial lag or phase shift in the system is negligible, the flow is an instantaneous indication of the pressure gradient. Thus, the total pressure gradient on the system is the linear addition of the pressure gradient arising from oscillation of the channel and the surface tension. These effects are $90^{\circ}$ out of phase and will be represented using sine and cosine functions. Eq \ref{equ:DeltaP} comes from the second derivative of a sinusoidal input to give acceleration. The acceleration is substituted into the equation for fluid pressure due to an acceleration field, $\Delta P = \rho \ddot{x} L$. 
%\begin{equation}
%\frac{\partial P}{\partial x} = -\frac{\Delta P_{osc}}{L}\sin(\omega t)  - \frac{Vol_{max}}{L} \frac{\partial P}{\partial V}\cos(\omega t)
%\label{equ:dPdx}
%\end{equation}
%\begin{equation}
%\Delta P_{osc} = A \omega^2 \rho L
%\label{equ:DeltaP}
%\end{equation}
%\begin{equation}
%Vol_{max} = \left| \int_{\omega t = \frac{\pi}{2}}^{\omega t = \pi} \left(-\frac{\partial P}{\partial x}\right)\frac{\pi R^{4}}{8 \mu} \partial t \right|
%\end{equation}

%The result is a circular definition for $\partial P/\partial x$. However, we can gain insight by calculating terms in succession. First we calculate the solution with only the first term in Eq \ref{equ:dPdx} (i.e. with no surface tension effects) to estimate a worst case displacement for $Vol_{max}$. It is a worst case because surface tension will generally work to reduce the displacment. We then substitute the result back into Eq \ref{equ:dPdx} to get Eq \ref{equ:dPdx_wc}, a worst case scenario for $\partial P/\partial x$. Notice that resonant frequencies are not of concern here in this viscous dominated system.
%\begin{displaymath}
%Vol_{max, wc} = \left| \int_{\omega t = \frac{\pi}{2}}^{\omega t = \pi} \left(\frac{\Delta P_{osc}}{L}\sin(\omega t)\right)\frac{\pi R^{4}}{8 \mu} \partial t \right|
%\end{displaymath}
%\begin{equation}
%Vol_{max, wc} = \frac{A \omega \rho\pi R^{4}}{8 \mu}
%\end{equation}
%\begin{equation}
%\frac{\partial P}{\partial x} \approx_{wc} \left(-A \omega^{2} \rho \right) \left(\sin(\omega t)  + \frac{R^{4} \gamma}{\omega a^{3} \mu L} \cos(\omega t) \right)
%\label{equ:dPdx_wc}
%\end{equation}

%So, we can see that if the coefficient of the cosine term (i.e. the ratio of the pressure gradients from oscillation and surface tension) is much less than 1, surface tension can be ignored in the solution of the system. To see this, let $A=500 \mu$m, $\omega = 20(2 \pi)$, $L=5$mm, and $a = R = 250 \mu$m. $\Delta P_{sin}/L$ has a magnitude of 7900 Pa/m. This results in $Vol_{max} = 1.54 \times 10^{-13}$ m$^{3}$ which, when multiplied by $\partial P/\partial V$ and divided by $L$, gives 229 Pa/m for the worst case pressure gradient due to surface tension. The coefficient of the cosine term is $\approx 0.029$. The flow rates of the solution with and without the cosine term is plotted in Fig \ref{fig:LoFreq}. Thus, the pressure gradient from surface tension and can be ignored. 

%\begin{figure}[!ht]
%\centering
%\includegraphics[width=2.5in]{OscPlot.pdf}
%\caption{Low frequency approximation to oscillatory flow in a microchannel. The solid line shows the system without surface tension perturbation while it is present in the red-dashed system.}
%\label{fig:LoFreq}
%\end{figure}

%It can be seen that very low frequencies are needed to make surface tension significant, especially for reasonably size ports. Ports are generally near the same radius as the channel. For the particular case shown in Fig \ref{fig:LoFreq}, the frequency would have to be reduce by a factor of 4 from a frequency of 20 Hz to 5 Hz. 

%The range of shear stresses that we are interested in is near 0.5 dynes/cm$^{2}$ per literature on sorting of cells in continuous shear flow. Preliminary vibration results with cells showed that frequencies somewhere in the range of 5-40 Hz are necessary to affect cell adhesion. However, at a frequency of 5 Hz, the estimated shear stress is 0.6 dynes/cm$^{2}$. Therefore, this might indicate that slightly higher shear is needed when using oscillation. This is quite speculative as the initial experimental setup was very undefined. The new setup has a defined acceleration and frequency.

%In conclusion, it appears that depending on what frequency is actually needed, surface may or may not need to be accounted for. If surface tension plays an important role, simulation will likely be the easiest route for estimating the shear environment of the cells.

%\section{Steady Flow Between Parallel Plates}
%With the origin positioned at the surface of the bottom plate (\ie , $y=0$) and the full height of the channel defined as $h$, the equation for $u$, the velocity in the x-direction is as follows.

%\begin{equation}
%\centering
%u= \frac{1}{2\mu}\left[\frac{d}{dx}(p+\rho g z)\right]\left(\left(y-\frac{h}{2}\right)^{2}-\frac{h^{2}}{4}\right)
%\label{equ:u}
%\end{equation}

%The flow rate through the cross section, where $b$ is the width that is $\gg h$ is as follows.

%\begin{equation}
%\centering
%Q=-\frac{bh^{3}}{12\mu}\left[\frac{d}{dx}(p+\rho g z)\right]
%\label{equ:Q}
%\end{equation}

%The average velocity of fluid between the plates is:

%\begin{equation}
%\centering
%\bar u = -\frac{h^{2}}{12\mu}\left[\frac{d}{dx}(p+\rho g z)\right] = \frac{2}{3}u_{max}
%\label{equ:Q}
%\end{equation}

%\begin{equation}
%\centering
%\tau_{wall} = -\frac{h}{2} \left[\frac{d}{dx}(p+\rho g z)\right]
%\label{equ:Q}
%\end{equation}

%\begin{equation}
%\centering
%\tau_{wall} = \frac{6\mu Q}{bh^{2}} = 2\mu \gamma_{wall}
%\label{equ:Q}
%\end{equation}

%\begin{equation}
%\centering
%\bar u = \frac{Q}{bh} = \frac{2}{3}u_{max}
%\label{equ:Q}
%\end{equation}

