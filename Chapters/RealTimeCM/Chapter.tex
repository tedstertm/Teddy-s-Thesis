\chapter{Assays: A Passive Microfluidic Real-Time Conditioned Media Assay}
\label{Chap:RealTimeCM}

\section{Preface}
I regret to disappoint those who are hoping for or expecting a simple and passive method for performing conditioned media experiments but, after coming quite close to success, I have come to understand a fundamental and critical limitation of our current approach. I still believe a real-time conditioned media assay is an important goal to pursue, but in a very different direction; a direction that I hope to convey here which builds on what I have learned up to this point. Given the nature of this chapter, reporting of results will be brief and focus will be placed on the `bottom-line'. Preliminary results from this project are also included in Appendix \ref{App:RealTimeCM} in a form similar to what was presented on this subject for my preliminary exam. It should also be noted that Erwin Berthier also contributed significantly to these efforts.

\section{Introduction}
Conditioned media experiments are a well established tool for establishing the \emph{presence} and \emph{directionality} of a soluble factor effect between two populations of cells. In a conditioned media experiment, fluid from one culture is transferred to another. Using this protocol, a key assumption can be made: the media is transferred in only one direction. In other words, there is no potential for the `downstream' population to influence the `upstream' population. Without the ability to make this assumption, the assay becomes a co-culture assay, for which there are many other tools available. This key point will become important later.

The original aim of this chapter was to use microscale control of flow to address one major drawback of the standard macroscale conditioned media assay; that drawback being the significant potential for soluble factor degradation. Protein half-life is an important intrinsic control mechanism of soluble factor signaling that can be influenced by cells via molecules to prevent or enhance degradation. In fact, half-life is one of the defining characteristics that differentiates hormones from other paracrine and autocrine factors. In conditioned media assays, the media is typically conditioned for hours or even days to accumulate the soluble factor(s) of interest. However, during conditioning of the media, factors are also degrading such that upon delivery to the receiving cell population, a significant portion of short half-life factors has already degraded. The effect can be significant given that some factors have half-lives on the order of minutes. The goal was to help reduce the effects of degradation by continuously delivering freshly conditioned media to the receiving cell type and to do it in a passive way that did not require any additional automation or other equipment. By conserving factors that are typically lost, assay sensitivity and repeatability are expected to increase.

The proposed approach required development of two passive elements, a diffusion valve to isolate communication between cell populations and a slow-flow pump for delivering conditioned media and operating the diffusion valve (Fig \ref{Chap:RealTimeCM:fig:oneWay}). A summary of the development of these components is provided; however, as it turns out, the use of open access ports will result in violation one of the key requirements of conditioned media assays: the compartments must remain completely isolated from one another. This is clarified in the sections below along with a discussion of a vastly different approach for achieving real-time conditioned media capabilities that offers more flexibility and potential for the development of new custom conditioned media assays.

\begin{figure}[!ht]
\centering
\includegraphics[width=3.5in]{OneWay.pdf}
\caption{\textbf{Microfluidic method for performing real-time conditioned media experiments}. Fluid from the upstream chamber is slowly passed to the downstream chamber via a diffusion valve (see Appendix \ref{App:DiffusionValve}) that acts to prevent communication upstream.}
\label{Chap:RealTimeCM:fig:oneWay}
\end{figure}

\section{The Diffusion Valve}
The diffusion valve is the long and narrow channel that connects the two culture chambers. Operation of the diffusion valve can be completely described using the Peclet number (Pe, Eq \ref{Chap:RealTimeCM:equ:Pe}). Pe describes the balance between diffusion and convection for transport of a particular solute. In the case of the diffusion valve, $L$ is the length of the diffusion valve, $v$ is the average velocity of fluid flow, and $D$ is the diffusion coefficient of the solute of interest in the fluid.

\begin{equation}
\textrm{Pe} = \frac{L\,v}{D}
\label{Chap:RealTimeCM:equ:Pe}
\end{equation}
 
Using a 1-D model of diffusion, the Pe number determines the concentration of factor that can reach the upstream end of the diffusion valve. Fig \ref{Chap:RealTimeCM:fig:Pe} plots concentration in a diffusion valve for various values of Pe. In this figure, fluid flow is from right-to-left, thus the upstream chamber is on the right (opposite of Fig \ref{Chap:RealTimeCM:fig:oneWay}).

\begin{figure}[!ht]
\centering
\includegraphics[width=3.5in]{DiffusionValve.pdf}
\caption{\textbf{Diffusion valve operation}. Plot of normalized concentration ratio (concentration at position $x/L$ divided by concentration at inlet of the downstream chamber) for various values of the Peclet number for the diffusion valve. $x/L=0$ is at the downstream chamber and $x/L=1$ is at the upstream chamber.}
\label{Chap:RealTimeCM:fig:Pe}
\end{figure}

Given how long and narrow the diffusion valve is, the 1-D model is expected to be quite accurate. For a given molecule of interest, if the Peclet number is 10 and the downstream chamber has a nominal concentration of 1, then the upstream end of the diffusion valve at steady-state would have a concentration of 0.00004. Thus, by tuning the flow rate of the device, varying degrees of isolation can be achieved between the up- and downstream chamber. Further discussion of the diffusion valve is  provided in Appendix \ref{App:DiffusionValve} and \ref{App:RealTimeCM}.

\section{Slow-Flow Pump}
The flow rates required to maintain a Peclet number of 10-20 are on the order of 10 \textmu L/day. Thus, very slow flow is needed to maintain isolation of the chambers while avoiding dilution of the culture microenvironments. Two different methods of slow flow were pursued, evaporation and gravity.

\section{Evaporation-Driven Slow Flow}
Evaporation can be used to drive fluid flow in passive-pumping based channels and has been described previously in the literature\cite{Frisk:2008pi,Walker:2002oy,Berthier:2008tl,Berthier:2008jf}. In order to drive fluid flow using evaporation, a relatively dry environment is needed, whereas a humid environment is needed for maintaining cell culture. Thus each evaporation design requires the use of a dry and humid environment. Two general ways were used to produce the dry and humid environments. The first was to utilize the humid environment of a standard culture incubator and to create an isolated dry environment \emph{on} the `chip'. The second was to use large dry environment like an un-humidified incubator, and to create a humid environment \emph{on} the `chip'. Whether the dry or humid environment was maintained on the chip, a method for controlling evaporation rate was needed. For this, an `evaporation vent' was designed that relied on 1-D diffusion to control escape of water vapor (see Appendix \ref{App:RealTimeCM}). The evaporation vent was an effective solution for controlling evaporation rates while it was much more difficult to design a humid or dry environment \emph{on} the chip.

\paragraph{On-Chip Dry Environment}
The vapor absorbing ability of silica gel was used to create a dry environment on-chip. However, after repeated attempts, it appeared that beads of silica gel were not able to result in sufficient evaporation through the evaporation vent. This embodiment was also tried using a PDMS evaporation barrier to increase evaporation but also met with mixed success.

\paragraph{On-Chip Humid Environment}
In this embodiment, the incubator is made dry while fluid is essentially sealed on the chip. The most effective method for doing this was found to be the use of a glass slide clamped to a polystyrene device and sealed using a silicone rubber gasket. There were a few drawbacks to this approach. The gas exchange is cut off between the incubator and the culture chambers. Also, as soon as the chip is taken out of the warm incubator environment, fluid from the channels begins to condense on the glass slide used to seal the humidity in. Thus, temperature gradients can induce rapid and dramatic evaporation even when the device is sealed. However, this embodiment was able to produce appropriate flow rates through the evaporation vent but was not ideal for culture and posed challenges with respect to temperature gradients.

Although evaporation could be easily metered using the evaporation vent, it was difficult to produce a robust humid and dry environment to drive flow. It may be possible to overcome some of these robustness issues; however, success in this regard will not fix the issues eluded to in the introduction that will be discussed later on. Thus, in the case of real-time conditioned media, evaporation methods are not worth pursuing even if a better method is developed. This was not apparent at this point in the development process, leading me to explore the use of gravity-driven flow as a means to avoid the many challenges of evaporation driven flow.

\section{Gravity-Driven Slow Flow}
Given the challenges faced using evaporation based methods, gravity driven flow was explored as an alternative. The concept was to use microchannel with sufficient resistance such that a large reservoir of fluid could be used to drive steady, slow flow through the device via gravity. This approach has some advantages over evaporation driven flow. The first is that a single humid environment can be used for the assay. This greatly reduces device complexity while remaining a passive device component. The second is that flow rate can be controlled by placing different volumes in the reservoir that drives flow. This is in contrast to evaporation where the evaporation vent or other barrier had to be altered to adjust evaporation.

This approach quickly led to successful slow flow. However one disadvantage of this method is that flow rate is dependent on the resistance of the microchannel device. For a channel with relatively flat cross-section, resistance is inversely proportional to the cube of the height. Thus, very small changes in channel height, even those observed within a single SU-8 master mold, can significantly affect the flow rate through the device. Still, manufacturing tolerances can be addressed using other device manufacturing methods. 

The most effective embodiment of a gravity driven flow device used the tube from a plastic syringe as the fluid reservoir. The tube was cut and sealed to the device using PDMS. The walls of the syringe tube are already very smooth and treated such that contact angles of aqueous solutions are near 90$^{\circ}$. These properties allow gravity-driven flow to remain relatively unaffected by surface tension and the meniscus that forms naturally within the reservoir.

After achieving success using gravity driven flow, validation experiments using dye quickly provided evidence of another issue that cannot be overcome without completely separating the culture chambers.

\section{A Limitation of Using Open Microfluidic Ports}
The open ports of a passive-pumping-based device act both as points of evaporation and as fluidic capacitors. The port acts as a capacitor by acting as a small reservoir for fluid while the fluid-air interface acts as a spring, attempting to keep pressure in the reservoir in balance with all the other connected ports of the device. In the case of the device illustrated in Fig \ref{Chap:RealTimeCM:fig:oneWay}, the ports that provide access to the culture chambers act like capacitors with a resistor between them (the diffusion valve). During slow flow, the pressure in the upstream chamber is almost identical to that of the downstream chamber except for the slightest difference that is necessary to drive 10 \textmu L of fluid per day. This is fine until the device is handled. Handling of the device typically results in some slight tipping of the device. By angling the device slightly, gravity drives flow from the port of one culture chamber to the other that can result in communication upstream, thereby violating the requirements of a conditioned media assay. 

Evaporation during handling can be many times faster than the desired perfusion flow rate (\ie, $\sim$ 10 \textmu L/hour instead of 10 \textmu L/day) resulting in violation of this requirement as well. On one occasion, fluorescent dye was placed in the downstream chamber of a gravity driven flow device. After 10 minutes on a microscope, intensity within the upstream chamber was roughly 3 - 4\% of the downstream chamber indicating a complete failure to isolate the chambers. It should be noted that because the device uses gravity to drive flow, the contaminating flow was strong enough to overcome the pressure of the driving reservoir. Given the desired perfusion rates of the device are so low, even the smallest of source of contaminating flow is likely to be significant. Further, due to the narrow nature of the diffusion valve, only nanoliters of flow are necessary to result in significant contamination.

Both of these effects result from the the fact that an open port acts as a capacitor and is unavoidable if passive-pumping is to be used to address the culture chambers. Although the assay may work if the assay is kept under extremely controlled conditions, it is unrealistic to expect significant use of a method with so many restrictions.

\section{Alternative Approaches for the Future}
Developments up to this point make it clear that if passive-pumping is going to be used to perform conditioned media experiments, \emph{complete separation of the culture chambers must be maintained}. The most obvious way to transfer fluid between two microchannel cultures is to use a pipette. This method has been suggested before by Ivar Meyvantsson \cite{Meyvantsson:2006gf}. This method has many potential advantages. First is that the assay can be automated and is thus completely programmable. Transfer rates can be tuned and do not rely on an device structures. The interactions of culture chambers are no longer limited to one direction and one flow rate. Complete isolation of signals can be maintained via the use of disposable pipette tips. The only remaining challenges for our laboratory would be to implement a liquid handler within an incubated environment and to minimize evaporation that could occur given that the use of a standard lid would prevent pipetting.

The potential flexibility and new types of assays, including real-time conditioned media experiments, that could be enabled using an incubated liquid handler provide significant motivation to address these challenges. Given the plethora of liquid handling automation and supporting equipment for manipulating trays and lids, it is likely that much of this system can be purchased, potentially leading to rapid development and experimentation.

