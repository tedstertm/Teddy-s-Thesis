%\chapter*{Conclusion}
%\addcontentsline{toc}{chapter}{Conclusion}
\chapter{Conclusion}
\label{Chap:conclusion}

Engineering for microfluidics for biological systems is not trivial. Engineers developing \invitro\ models have a lot of experience now and are capable of creating highly relevant constructs with many components that can be finely tuned to explore the biology of the model. A good design that enables new biology, faster biology or better biology should be embraced. Unfortunately, too many enabling technologies end up as a single publication in an engineering journal and sit in a wafer box on a shelf. One of the biggest hurtles in getting microfluidic platforms to achieve their full potential moving them into the hands of researchers who are capable of asking relevant, and impactful questions of them. The engineers who design these models have wondered why biologists are not lining up to get their hands on potential game-changing technologies, there's a sizable body of literature analyzing what engineers are doing wrong and how they can change to attract broader interest to their research. This document adds to that body of literature in hopes that something would stick and make microfluidic-enabled \invitro\ models somewhat more relevant to biologists. 