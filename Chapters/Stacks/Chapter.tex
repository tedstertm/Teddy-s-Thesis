\chapter{Stacks for the multiple myeloma microenvironment}
\label{Chap:Stacks}

\section{Introduction}
Let's talk about how great stacks is ok.
\section{Methods}
Filler

\subsection{Fabrication}
We used a mill, then we used injection molding, polystyrene was used and then polypropylene.

\subsection{Device operation}
Filler

\subsection{Multiple myelolma culture}
Filler

\section{Results and discussion}
Filler 

\subsection{Stacks physics}

Discuss spontaneous capillary flow here. At first we used polystyrene then we switched to polypropylene since the conditions for spontaneous capillary flow are dependent on surface contact angle, and the surface contact angle of water any polystyrene is much lower than that of water and polypropylene.

The distribution of fluid in stacks should behave similarly to double-sided open well \cite{DeGroot2016}. At lower volumes, fluid is evenly distributed between the top and bottom of the well. As more fluid is added and the mass of the system increases, fluid is pulled downwards in the direction of gravity and becomes elongated \cite{Carvajal2011}, the end result is a well with a small sessile drop on the top side, and larger pendant drop on the bottom that makes fluid replacement from the top difficult with only one stacks layer. A solution to this issue is to use an additional stacks layer placed on top as a means to change media. The layer fluid exchange layer would be filled with media and then placed on top of the stacks culture where fluid exchange happens through diffusion. Additional fluid exchange steps require the removal of the exchange layer from the assembled device, replacing the fluid in the exchange layer and reassembling the device. We found that slightly altering the geometry of the device eliminates the need for an exchange layer in many cases. If the diameter of the well is slightly bigger at the top of the well than at the bottom it will lower the relative pressure of the droplet on top compared to that on the bottom at low volumes. As volume is increased in this system the one of the principal radii of the larger droplet will continue increasing relative to the bottom droplet further decreasing pressure in the top droplet resulting in fluid added to the system collecting on top of the stacks device or assembly instead of at the bottom. Simply having a slightly larger radius on the top of the device than on the bottom facilitates fluid exchange on top of the device without the necessity for an additional fluid exchange layer.