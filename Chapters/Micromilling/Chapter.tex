\chapter{Micromilling: a method for ultra-rapid prototyping of plastic microfluidic devices}\footnote{This chapter has been modified from the published manuscript of the same title. The manuscript includes as authors John Guckenberger, Theodorus de Groot, Alwin Wan, David Beebe, and Edmond Young}
\label{Chap:Micromilling}

This tutorial review offers protocols, tips, insight, and considerations for practitioners interested in using micromilling to create microfluidic devices. The objective is to provide a potential user with information to guide them on whether micromilling would fill a specific need within their overall fabrication strategy. Comparisons are made between micromilling and other common fabrication methods for plastics in terms of technical capabilities and cost. The main discussion focuses on "how-to" aspects of micromilling, to enable a user to select proper equipment and tools, and obtain usable microfluidic parts with minimal start-up time and effort. We aim to reach an audience with minimal prior experience in milling, but with strong interests in fabrication of microfluidic devices.

\section{Introduction}
The microfluidics community now has an enormous selection of materials, methods, and techniques for developing microfluidic systems. Many different techniques and methods are now readily available, and are becoming increasingly accessible as a result of heightened research efforts and growing interest in the field. Such a diverse repertoire of methods and tools increases the potential for the advancement, adoption, and proliferation of microfluidic technologies, and opens new avenues for research and development in both academia and industry. 

Of all the materials commonly used in microfluidics, plastics remain a primary option due to their many favorable properties and its compatibility for biology applications (e.g., polystyrene is commonly used for mammalian cell culture) \cite{Young2011}. Plastics are low cost and highly amenable to high-volume manufacturing processes, making them particularly suitable for those who are developing technologies for commercialization and mass production \cite{Mukhopadhyay2007}. For these reasons, plastics have been considered a reliable and robust material since the early years of microfluidics, even as other materials such as PDMS and paper have become increasingly popular \cite{Becker2000a, Berthier2012}. With microfluidics entering its third decade, and garnering heightened interest from industry, plastics -\, and their related fabrication processes \hyphen\, will likely play a major role in translating microfluidics research into commercialized technologies \cite{Sackmann2014}.



Many fabrication methods for plastics are available to researchers, with each fabrication method offering different advantages and limitations \cite{Mukhopadhyay2007}. Some methods such as injection molding have existed for decades, and are well studied, but have high start-up costs that limit their utility for low-volume production. Other methods such as laser micromachining \cite{Sugioka2010, Wang2006, Gattass2008} stereolithography \cite{Au2014, Waldbaur2011} are rapidly evolving due to ongoing advancements in technology, and are thus not as well studied as other traditional methods \cite{Au2014}. While the current collection of fabrication methods can meet a wide range of technical needs, various gaps still exist within the area of microfabrication that are difficult to address with only these most common methods.

Micromilling is an alternative method that has the potential to address some of the challenges in microfabrication. Micromilling is a fabrication method that creates microscale features via cutting tools that remove bulk material. While many other methods have been discussed previously for microfluidics applications \cite{Becker2000,Becker2000a}, micromilling has received much less attention. Recent work has shown micromilling to be effective for microfluidic devices \cite{Kummrow2009, Wilson2011}. For example, Kit-On-A-Lid-Assays (KOALA), designed to deliver fluid by assembling multiple slides that can be clipped together, have been used for a variety of assays \cite{Berthier2013, Guckenberger2015}, and are fabricated by micromilling. Micromilled devices have also been used to create oil and aquaeous interfaces for cell capture and RNA, DNA, and protein isolation \cite{Casavant2013a, Strotman2013}. Bischel and co-workers utilized milled devices to capture and orient zebrafish for imaging and drug testing \cite{Bischel2013}. Carney and co-workers captured primary fetal testis cells in mixed and comparmentalized co-culture to study microenvironmental factors regulating steroid production and organ co-culture assays. However, the method remains underutilized for microfabrication compared to other methods. This is largely because of presumed high start-up costs, the need for large equipment and lab space, and the need for unique technical expertise. However, recent developments in machining technology have alleviated many of these drawbacks, making micromilling a potentially important option in microfluidics. 

In this tutorial, we present micromilling as a microfabrication method for plastics, and provide practical tips and strategies for achieving ultra-rapid prototyping of microfluidic devices. First, we compare costs and capabilities between micromilling and other common plastic microfabrication methods, allowing the practitioner to determine whether micromilling is suitable for the target application. Second, we present an operational guide to provide insights on how to best set up the machine and tooling, and to select appropriate parameters to enable optimal, reliable, and precise fabrication of microfluidic devices. Third, we address questions related to the quality of micromilled devices by measuring and comparing surface roughness, precision, and repeatability, as produced by micromilling (with a low cost milling system), and by other methods. As a demonstration of the utility of micromilled plastic devices, we culture mammalian cells in milled microchannels to characterize cell viability and image quality under typical conditions required for conducting cell-based experiments. Lastly, we discuss the advantages, limitations, and overall potential of micromilling as a useful fabrication method. This tutorial will serve as a guide for practitioners who are considering using micromilling in their fabrication repertoire.


\section{Fabrication overview}
A critical step in developing microfluidic systems is choosing the proper microfabrication method. In this section, we provide the microfluidics practitioner with a basis for considering micromilling as a useful fabrication method, particularly for ultra-rapid prototyping of microfluidic devices in plastics (and potentially other materials). First, background and fundamental aspects of micromilling are described. The strengths and limitations are then summarized and compared (in terms of technical capabilities and affordability) to three other common plastic microfabrication techniques: hot embossing, injection molding, and 3D-printing via stereolithography.

\subsection{Micromilling}
Milling is a subtractive manufacturing process that uses rotating cutting tools to remove material from a starting stock piece, commonly referred to as the workpiece. The basic milling system, or mill, consists of (1) a worktable for positioning the workpiece, (2) a cutting tool (most commonly an endmill), and (3) an overhead spindle for securing and rotating the cutting tool (Fig. \ref{figure:MillFig1}). Milling, which has origins dating back to 1818 \cite{Woodbury1974}, has undergone significant advances, and now represents a major tool in a machinist's repertoire. The positions of the worktable (X and Y-axis) and spindle (Z-axis) are traditionally adjusted by hand with mechanical levers and cranks, but modern mills now employ computer numerical control (CNC) that automates the process, thereby improving repeatability and precision, reducing human error, and adding advanced capabilities (e.g., the direct conversion of computer-aided design (CAD) models to finished parts). 

Milling machines with CNC capability (i.e., CNC mills) are available with a wide range of technical specifications, encompassing varying levels of stage precision, spindle speeds, and automation. Modern CNC mills are versatile and capable of fabricating devices with features ranging in size from several microns to several meters \cite{Auric2012}. The wide availability of cutting tool shapes, materials, and sizes \cite{Friedrich1996a} makes the mill amenable to fabricating many types of features in many different materials. Perhaps the most enabling aspect of using a CNC mill is the ability to fabricate a part directly from a three-dimensional (3D) CAD model, making it easier and faster to convert design concepts to working prototypes. Latest advances in technical features have enabled improved precision and resolution down to the micron scale, leading to the use of the term micromilling to describe fabrication of increasingly more intricate parts with microscale resolution \cite{Bang2005}. 

Micromilling can be useful in microfluidics applications: for two main functions (1) machining the mold used in subsequent fabrication steps (e.g., embossing or injection molds) \cite{Wilson2011,Okagbare2010}, or (2) machining microchannels and features directly into the final part. In the latter case, micromilling offers a key advantage: a plastic workpiece can be milled into a device in less than 30 min, significantly reducing turnaround time from design to prototype. 

Milling is well characterized for producing large features in common machining materials such as steel and aluminum. Thus, technical information is available from machining handbooks, shop technicians, and online resources \cite{Chi-Hsiang2013}. In addition, comprehensive reviews are available on multi-functional machine tools for metal cutting \cite{Moriwaki2008}. In contrast, milling of microscale features in non-traditional milling materials such as plastics is much less characterized, especially in the context of microfluidic devices \cite{Chen2014}. Thus, there is a need to fill this gap in technical knowledge to determine the usefulness of micromilling in microfluidics.

\begin{figure}[h!] %DONE
\centering
\includegraphics[width=3in]{/MillFig1.jpg}
\caption[\textbf{Milling overview}]{(A) A schematic showing the basic components of a CNC mill, which can use computer-aided design (CAD) models to produce finished devices. The mill consists of a worktable (to provide motion in the XY-plane), a cutting tool (to remove material from the workpiece), and a spindle (to hold the cutting tool, spin the cutting tool, and provide motion along the Z-axis). (B) A photograph showing A CNC micromill during operation. (C) A photograph of a micromilled thermoplastic device that contains a variety of feature geometries and sizes.}
\label{figure:MillFig1}
\end{figure}

\subsection{Other fabrication methods for plastics}
To facilitate the discussion of micromilling in the context of microfluidics, we compared it to three of the most commonly discussed microfabrication methods for plastics: (1) injection molding \cite{Attia2009, Tanzi2013}, (2) hot embossing \cite{Becker2000a, Abgrall2007}, and (3) stereolithography \cite{Waldbaur2011, Au2014}. All three methods have been reviewed elsewhere \cite{Melchels2010}. For convenience, key aspects of these methods are summarized here.

\textit{Injection molding} is a process in which molten polymer is injected into a mold cavity (often made of steel or aluminum) that contains the template for the features desired on the finished part. The molten polymer conforms to the features within the cavity, hardens during cooling, and is then ejected from the mold to yield a finished part. \textit{Hot embossing} also involves conforming molten polymer to a mold, but instead of injecting the polymer into a cavity, the polymer is pressed against the mold at high temperature and pressure to transfer the desired features from the mold to the softened polymer. While injection molding and hot embossing are indirect fabrication methods that create parts using molds (i.e., intermediary parts that require their own fabrication), other methods can offer more direct approaches that do not require molds or other ancillary components. \textit{Stereolithography}, for example, is an additive manufacturing process that builds 3D parts one layer at a time by curing photosensitive polymeric materials with a laser or other light source, and thus does not require a mold. Classified as a 3D printing method, along with stereolaser sintering and extrusion deposition modelling \cite{Waldbaur2011}, stereolithography is rapidly gaining in popularity, and has demonstrated potential for significant utility in microfluidic applications \cite{Au2014,Morimoto2009, Bhargava2014}. Note that while milling and stereolithography are both direct fabrication methods, they differ in that milling is a subtractive process, whereas stereolithography is an additive process.

\subsection{Technical comparison}
A fabrication method must first and foremost offer the technical capabilities required to create the part. Three main technical factors need to be considered: (1) compatibility of that method with the material of choice, (2) ability to achieve the desired features, and (3) quality of the finished part. Since some methods are only compatible with certain materials, the method and the material must be chosen together to achieve optimal quality in the finished product. Since micromilling offers some unique technical capabilities, we were interested in comparing it to other common microfabrication methods to see whether its strengths could be leveraged for microfluidics. Technical capabilities of micromilling and three other microfabrication methods were tabulated, along with their compatibility with different materials and their ability to achieve certain geometric features (Fig. \ref{figure:MillFig2A}). 

\textbf{Material Compatibility}. Hot embossing and injection molding are suitable for polymeric materials, but are impractical for more brittle, rigid materials like glass and metal because both methods create significant stress on the material that often leads to cracks and other defects. Hot embossing can be applied to glass if the temperature is high enough, or to metals if conformable thin films are used \cite{Kumar2009, Rabe2007}. In contrast, stereolithography uses photo-curable polymeric resins, which have similar properties to various common polymers, but also have important differences. For example, Accura 60 (3D Systems Inc, Rock Hill, SC) is marketed as an analog to polycarbonate in terms of transparency and stiffness, but it differs in terms of hydrophilicity \cite{Waldbaur2011}. Micromilling is a primary method for machining metals \cite{Childs2000}, but is also amenable to plastics (as we show in this review). Milling of elastomers is difficult due to large elastic deformation that prevents effective material removal \cite{Shih2004}. In addition, glass and ceramics, which are generally difficult to fabricate, are difficult to mill because their brittleness leads to chipping, cracking, and other defects.

\textbf{Feature Capability}. Micromilling and stereolithography provide the widest range of feature capabilities with the least added process complexity. Stereolithography is capable of making complex 3D features that may be impractical or unfeasible with other methods. Complex features such as undercuts, rounded surfaces, and sharp internal-corners (see below for examples) can be achieved by micromilling using specialized tooling and extended process times. Since hot embossing and injection molding both rely on molds for processing, the aspect ratios and feature resolutions are determined by the quality of the molds themselves. Because the molds are often fabricated by milling, the part is no more complex than that achieved by direct milling. Mold quality can, however, be improved through downstream processing of the mold by alternative fabrication methods, such as soft lithography replication \cite{Young2011}. For injection molding, complex features (e.g., overhangs) are enabled via additional molding mechanisms (e.g., cams). Note that feature capabilities, especially aspect ratios, certainly depend on the properties of the materials used (e.g., melt flow, ductility, and toughness), and thus, feature capability and material choice go hand in hand. 

\textbf{Quality of Finished Part}. Quality of the finished part may be assessed by surface roughness, replication fidelity of features, and optical characteristics after fabrication. Surface roughness from hot embossing or injection molding originates from the mold itself, and, if necessary, can be improved significantly through mold polishing. This can be costly and time consuming, especially if multiple designs are being tested. Surface roughness from stereolithography, in contrast, depends on voxel (i.e., 3D-pixel) size and scanning resolution, and often differs significantly depending on orientation of the device within the printer. For micromilling, surface roughness depends on the cutting tool (e.g., tool features, profile, and wear), and the operational parameters chosen for the tool. These parameters can be optimized to reduce surface roughness, as described below.

Besides surface roughness, feature transfer fidelity is another important metric of quality. To ensure that the final part has dimensions that match the original design, one must account for mold shrinkage, a known effect where parts thermally "shrink" as a result of cooling. This issue is avoided when direct approaches such as milling and stereolithography are employed.

Finally, in terms of optical characteristics, hot embossing of optically transparent polymers can increase autofluorescence of the material \cite{Young2012}; this is a result of the rearrangement of polymer chains. The increased autofluorescence becomes important for applications involving fluorescence detection and microscopy. Interestingly, milling does not change the bulk molecular orientation of polymer chains within the plastic material the way it would change under high temperature processes like hot embossing, and as such, increased autofluorescence is avoided.

\subsection{Cost comparison}
A second major consideration when choosing the appropriate fabrication method is the overall cost of employing the method, including equipment, operational, and production costs (Fig. \ref{figure:MillFig2}B). In both academia and industry, the operational cost and potential return on investment eventually become key factors in determining the choice of method, as long as the final part is fabricated to the desired specifications and with sufficient quality to ensure functionality.

First, the required equipment and infrastructure costs for the process will determine the feasibility of simply acquiring the equipment and fabricating the parts in-house. In the case of soft lithography, for example, many universities already have central clean-room facilities that were previously established for research in fields such as microelectronics and micro-electro-mechanical systems (MEMS). This infrastructure provides microfluidics researchers access to the required equipment for lithographic processes, which enabled rapid adoption of soft lithography without the need for new infrastructure.

Injection molding has the highest start-up cost based on infrastructure, equipment setup, and technical know-how for producing quality parts. For this reason, injection molding is commonly outsourced, with a growing number of companies providing injection molding services tailored for microfluidics (e.g., Microfluidic ChipShop, Micronit, SimTech, thinXXS, Symbient). Even if one chooses to outsource, there remains the cost of expensive molds that must be made specifically for each design. This quickly becomes costly during design iterations, and thus, injection molding is typically reserved for mass production of the finalized design. In contrast, start-up costs for hot embossing, milling, and stereolithography processes can be much lower than for injection molding. Entry-level heated presses, milling machines, and stereolithography printers can cost less than \$15,000, which helps those working with a tight budget. However, as is often the case, low cost usually means fewer technical features, and greater limitations on spatial resolution, size of workspace, printing speed, material compatibility, and automatic control. 

In addition to the cost of the main equipment, accessories can add significantly to operational and maintenance costs of a fabrication method. Milling requires purchasing tooling in the form of endmills that need to be periodically replaced due to damage or wear. While this adds to the list of regular supplies needed for milling, this cost is still significantly lower than the cost of molds required for each design when either injection molding or hot embossing are employed. Furthermore, molds are not only expensive, but require long lead times for their own fabrication. When both the costs of parts and labor are considered, it is understandable that both hot embossing and injection molding are reserved for medium- to high-volume production, respectively. In contrast, operational costs associated with stereolithography and other 3D-printing methods correlate to the material volume of the part (i.e., the amount of material used), as opposed to the design complexity. Thus, stereolithography, because it is an additive process, becomes highly attractive for complex and intricate objects; it is specifically advantageous for parts that use little material or are otherwise difficult to manipulate with other processes, such as micromilling. 

A crucial factor that justifies the equipment and operational costs of more expensive methods is the required quantity of production. If higher-volume production is required, injection molding and hot embossing become increasingly more cost effective per part, particularly since these processes can often be parallelized to further increase throughput. In contrast, direct fabrication methods like milling and stereolithography are generally serial fabrication processes, and therefore the cost per part remains fairly constant, independent of the quantity produced. As a result, milling and stereolithography become economical for rapid design iterations, because there is no need for molds or other intermediate production steps.

\begin{figure}[h!] %DONE
\centering
\includegraphics[width=5.5in]{/MillFig2.jpg}
\caption[\textbf{A comparison between milling and other microfabrication methods for plastics}]{\textbf{A comparison between milling and other microfabrication methods for plastics, in terms of:} (A) material compatibility, feature capability and quality; and (B) cost. In (A), three filled circles = "excellent", three open circles = "impractical" or "inadequate"; see legend (bottom left of A) in (B), for process times, "Time" represents the time of fabrication for one device for both on-site and outsourced devices. "cost" (in USD) is an estimate, where on-site fabrication is calculated based on the cost of goods used (not including labor; estimated from the labs of other authors), and outsourced fabrication is based on the lowest quoted price we obtained for different quantities. N/A = not applicable.}
\label{figure:MillFig2}
\end{figure}

\subsection{Summary}
We compared micromilling to three other fabrication methods -- stereolithography, hot embossing, and injection molding -- and assessed their technical capabilities and overall costs. Each method has advantages and limitations, but micromilling offers unique advantages when it comes to ultra-rapid prototyping because of its low start-up cost, high resolution, and versatility regarding feature geometries and material choices. While there may be challenges with employing this method, these can be overcome with proper selection of equipment, setup, and alignment. To this end, the following sections discuss in detail how these challenges can be addressed so that the benefits of micromilling can be leveraged for microfluidics applications.


\section{Equipment}
Once the fabrication method has been chosen, the next step involves selection of the appropriate equipment to meet the needs of the user. For micromilling, choosing the appropriate equipment is critical to ensuring the desired quality of the end product. Milling equipment can be divided into two major categories: (1) the CNC vertical machining center (the central unit that drives the milling process), and (2) the tooling, which comprises the interchangeable parts and accessories that are attached to the central unit. In this section, we discuss the main criteria for selecting an appropriate mill system and associated tooling, with a particular emphasis on microfluidics applications.

\subsection{CNC mill systems}
CNC vertical machining centers, more commonly referred to as mills, are available in a wide range of configurations that vary in their technical specifications and cost (Fig. \ref{figure:MillFig3}A). With many available options ranging from affordable entry-level systems to advanced, high-precision systems, choosing the right mill for an application can be complicated. Most CNC mills are defined by the following features: (1) work envelope, the region of space defined by the allowable motion in the X, Y, and Z directions; (2) feed rate, the translational speed of the stage; (3) spindle speed, the rotational speed of the spindle that holds the cutting tool; (4) power provided by the motor, which often depends on speed and is used to determine appropriate spindle speeds and feed rates for a machining process; (5) the automatic tool changer (ATC), a mechanism that automatically changes cutting tools during a milling process, thus eliminating user intervention; and (6) precision, or the minimum cutting tolerance achieved by the mill. The \textit{accuracy} of the mill depends on several factors categorized by geometrical errors and wear, kinematic errors, thermal errors, and stiffness errors \cite{Lamikiz2008}. Amongst these mill characteristics, accuracy and speed usually impact cost the most. For example, a basic CNC-milling machine (with accuracy of <25 \textmu m) can be obtained for \textasciitilde\$15,000, whereas, a milling machine with automated tool alignment (with accuracy of <3 \textmu m) may cost over \$200,000. In general, CNC-mills can be customized with features that increase throughput and accuracy, albeit at a higher cost. CNC mills that cost over \$100,000 often have integrated systems to align and position the workpiece and cutting tool.  Thus, different systems and options are available for almost any budget.

\subsection{Endmills}
The most common tool for milling is the endmill. Endmills remove material by cutting along any axis (i.e., X, Y, or Z), and are commercially available in various sizes, shapes (also called profiles), and materials (Fig. \ref{figure:MillFig3}B) \cite{Kim2008}. Endmills have sharp helical grooves, or \textit{flutes}, that wind from the tip of the endmill toward the shaft. The appropriate number of flutes and their helical angles should be selected based on the application and the material to be cut. Selecting endmill size is straightforward, and depends largely on the desired feature dimensions and their desired resolution. Selecting the appropriate profile and material can be more challenging, however, because of the vast array of available options. The combination of size, shape, and profile significantly affects the dynamics and potential errors at the cutting edges \cite{Jun2006}. Square (cylindrical) and ball endmills are "workhorse" profiles that can be used for simple flat features, while ball endmills can be used to mill additional 3D features, including filleted corners, tapered edges, and contoured features \cite{Wilson2011}. Bull-nose and tapered endmills can also be used to create filleted corners and tapered edges, respectively, and reduce the required cutting time compared to a ball endmill. Many other profiles can be purchased or custom-manufactured for specific applications. Hole machining can be easily performed with square endmills, but drill bits (which only cut in the vertical Z-direction, unlike endmills) are often preferred for holes with a depth-to-diameter ratio greater than 3:1. Undercut features can be produced with woodruff cutters in a secondary mac hing process, after a slot or edge has first been created with one of the previous tools. Certain geometries, such as undercuts and threads, remain challenging even with micromilling, but overall, the wide variety of available endmill profiles allows 3D surface contours to be produced more directly, enabling the creation of microfluidic features with more complex topography that would otherwise be costly, time consuming, or even impossible to produce with other microfabrication techniques.

High-speed steel and carbide are the most common endmill materials, with carbide more commonly used for micro-endmills. Various endmill coatings are available to increase strength and lubricity, promote the removal of chips (i.e., the small pieces of material cut from the workpiece), and increase resistance to both heat and wear, in order to extend the life of the tool. Thus, coatings should be carefully considered and selected when machining tough materials, such as stainless steel \cite{Aramchareon2008, Endrino2006}, but are less critical for softer materials, such as plastics, where less heat and wear are produced. 

Endmills have several dimensional and physical characteristics that should be considered beyond profile and material (Fig. \ref{figure:MillFig3}C). Flute length and cutting diameter directly determine the maximum cutting depth and minimum width of a microchannel, respectively. Though it is possible, it is generally not advisable to mill deeper than the flute length of a given endmill (often three times the diameter); special extended-reach endmills are available if deeper channels or taller features are necessary, but they are generally more expensive, and flex more due to their length, increasing their likelihood of breaking. The shank is the part of the endmill that is inserted into the milling machine; thus, the shank diameter must match the size of the tool holder or \textit{collet}. The helix angle facilitates removal of chips. Insufficient chip removal will lead to clogging of the flutes, which will create heat, and ultimately damage the device, endmill, or both. When milling plastics and other soft materials, heat build-up can be particularly problematic, so lower helix angles are preferred (30\textdegree \, is the industry standard) because they provide more space between flutes, thereby improving chip removal. Additionally, the number of flutes can affect chip removal. For most plastics, two-flute micro-endmills can be operated at faster cutting speeds than four-flute endmills because they allow for better chip removal. On the other hand, four-flute endmills yield lower surface roughness than two-flute endmills. Lastly, the center-cutting classification of endmills indicates whether the endmill is capable of cutting (plunging) in the Z-direction, like a drill bit. Non-center-cutting endmills maintain the capability of cutting in the Z-direction, but must simultaneously move in the XY-plane. This is most often achieved by cutting with a spiral or ramp-like motion. Overall, careful consideration of the endmill characteristics discussed above, including size, material, and profile, will enable the user to select appropriate endmills, and produce high quality parts.

\begin{figure}[b!] %DONE
\centering
\includegraphics[width=5.5in]{/MillFig3.jpg}
\caption[\textbf{Milling center and tooling comparison}]{(A) CNC mills from several manufacturers are compared and categorized into price ranges. Costs were assessed based on quotes of the lowest level mill from each manufacturer, except the Tormach mill, which was quoted to be comparable in terms of capabilities to the other CNC mills. Unlisted specifications were not given by the manufacturer. (B) Endmills -- the most common cutting tool for milling -- are available in many profiles, in a variety of materials, and with a variety of coatings. Mills are also compatible with a variety of other cutting tools, some of which are shown. (C) Endmills are defined by several characteristics, each of which contributes to the endmill capabilities and feature quality.}
\label{figure:MillFig3}
\end{figure}


\section{Quality comparison}
When considering micromilling for fabricating microfluidic devices, an obvious concern is how the quality of milled parts compares with parts fabricated via other methods, given the surface roughness that typically results from the milling process. With the proper CNC mill setup and operation, however, it is possible to achieve sufficiently high resolution and low surface roughness, allowing micromilled devices to be used for a variety of applications, including cell-based assays.  In this section, we provide an evaluation of the quality of microdevices milled with an entry-level CNC milling machine. Specifically, we (1) detail the setup and procedures that we utilize to mill high-precision microdevices; (2) assess the resolution and surface roughness that are achieved with our milling process; and (3) assess the utility of milled channels for cell culture studies. For comparison, we assess the quality of hot-embossed microchannels fabricated using methods described by Young \textit{et al.} \cite{Young2011} Hot embossing was chosen for comparison based on our previous observations of low surface roughness, and based on cost, turnaround time, convenience, and direct compatibility with cell culture, which are all more comparable to micromilling than for injection molding or stereolithography. 


\subsection{Setup and procedures}
Proper setup and procedural steps can contribute significantly to improving the quality of the milled part. In particular, workpiece fastening and tool alignment are perhaps the two most influential factors on quality.For workpiece setup, we have achieved the best final quality by using adhesive tape to secure the plastic sheets to a flat granite block. We recommend securing the block to the worktable using strap clamps located at each of the four corners, or two corners and an opposing center (triangular setup).  These clamps are then adjusted to level the block, using a drop test indicator for verification. For our study, the block was levelled to $\pm$0.00025" (6.4 \textmu m) across a span of 10" (254 mm) along the X-axis and 5" (125 mm) along the Y-axis. Plastic sheets are secured to the block using an adhesive. If the adhesive is difficult to remove from the workpiece, a protective sacrificial (easy to remove) tape can be applied to the workpiece. The flatness of the plastic ideally mimics that of the block. However, clumps in the tape, unwanted debris, and air bubbles must be avoided, as these artifacts lead to localized height variations in the plastic sheet.


\subsection{Surface roughness and resolution}
Surface roughness and resolution are important metrics for assessing the quality of microfluidic devices, particularly for cell biology studies that require microscopy, or that utilize surface interactions (e.g., microfluidic ELISAs with substrate-bound antibodies), where roughness can impact proper operation and control. To this end, we measured the surface roughness of microchannels milled into PS \cite{Young2011, Chen2008a}, PMMA \cite{Wabuyele2001, Klank2002},and cyclic olefin copolymer (COC) \cite{Steigert2007}, three transparent polymers commonly used for microdevice fabrication (Fig. \ref{figure:MillFig4}A). In general, the surface roughness is proportional to the feed rate (i.e., surface roughness decreases as feed rate decreases). We expected surface roughness to vary inversely with spindle speed based on other studies \cite{Kiswanto2014}; Instead, we observed the lowest surface roughness at a spindle speed \textasciitilde5000 RPM. We believe that this may be due to the limits of our particular CNC mill, and that system-specific vibrations \cite{Zhang2007} mnay influence the optimal spindle speed for minimizing surface roughness. It is therefore likely that the surface roughness data will be different on mills from other manufacturers, and practitioners should be careful to conduct their own tests to determine optimal spindle speeds. We found that the depth of cut (up to 300\% had minimal effect on the surface roughness.

A second major concern associated with surface roughness is the presence of burrs -- small chips of plastic that remain attached to the workpiece after machining \cite{Lee2005}. Burrs occur most commonly in ductile plastics (e.g., COC or polypropylene) and are often prevalent along faces and edges. Occasionally, burrs will form in brittle plastics (e.g., PS or PMMA), but are often limited to corner edges, as opposed to faces. There are several simple considerations for reducing the likelihood of burrs \cite{Dimov2004}. The first consideration is to properly choose one of two opposing directions of cutting, referred to as \textit{conventional (up) milling and climb (down) milling} \cite{Toh2004}. Conventional milling is characterized by the workpiece moving directly against the cutting teeth of the endmill at the point of contact, such that the endmill adds resistance to the workpiece motion. In contrast, climb milling represents the workpiece moving in the \textit{same direction} as the rotating cutting teeth at the point of contact, as if the endmill was reducing resistance to the workpiece motion, and "climbing" along the workpiece surface in the direction of travel. For non-brittle plastics (e.g., polypropylene), conventional milling is recommended to minimize burrs. However, it should be noted that climb milling can also produce high quality finishes for metals \cite{Bernardos2002} and more brittle plastics like PS and PMMA. The second approach is to ensure that the tool is sharp. Dull tools, especially when used with low chip loads (i.e., high spindle speed or low feed rate), cause high levels of friction and generate heat. Heat, in turn, increases material ductility resulting in burrs, or in the worst case, can lead to melted plastic and tool breakage. For these reasons, it is important not to use feed rates that are too low, or spindle speeds that are too high, especially with dull endmills. The third approach, especially if burrs are present on vertical corners, is to adjust the toolpath to avoid tool exits. Many of these factors are changeable settings in computer aided modelling (CAM) software software packages and should be tailored to achieve the best quality possible.
When considering milling resolution, it is important to note that while mill tolerances are often specified by the manufacturer, they can also depend on setup and operational parameters. For this reason, we assessed resolution in terms of both accuracy (i.e., ability to achieve a target dimension) and precision (i.e., consistency across features, or low variability) using our mill and workholding techniques, and compared the results from milling to those from hot embossing. Resolution in the XY-plane is unaffected by misalignment, and dependent only on the technical capabilities of the mill (Fig. \ref{figure:MillFig4}B). Using our CNC mill with parameters that yielded the lowest surface roughness (from Fig. \ref{figure:MillFig4}A), we consistently achieved <0.001" (<25 \textmu m) accuracy, as expected, based on manufacturer specifications. Importantly, tool flexion induced through factors such as higher chip loads, increased depths of cut, and direction of cut, can lower accuracy. Experienced machinists often employ a final (low chip load) finishing cut to improve accuracy and further reduce surface roughness. In the Z-axis direction (i.e., for feature heights), the milling is repeatable between each separate channel (Fig. \ref{figure:MillFig4}C). The large variation (\textasciitilde.001") observed between the target height and the measured height was a result of tool alignment, in the Z-axis, to the workpiece.  This variation can be reduced by using smaller step distances, by using magnifying optics to better observe tool contact, or alternative (e.g., automated/electronic) approaches for tool alignment. 

These results demonstrate that with the appropriate setup and operation, the surface roughness, accuracy, and precision achieved by micromilling are in fact comparable to those achieved by hot embossing (Fig. \ref{figure:MillFig4}B-C). One important difference between milling and embossing, however, is that sharp internal corners are difficult to fabricate via milling because the endmill inherently creates an internal radius of curvature (Fig. \ref{figure:MillFig4}D). However, this may not be an issue for many applications.

\begin{figure}[ht!] %DONE
\centering
\includegraphics[width=4.25in]{/MillFig4.jpg}
\caption[\textbf{Surface roughness and resolution using an entry-level CNC mill.}]{\textbf{Surface roughness and resolution using an entry-level CNC mill.} (A) Colored contour plots showing surface roughness as a function of spindle speed (y-axis) and feed rate (x-axis). Surface roughness was measured by interferometry , and ranged from 0.420 to 1.52 \textmu m (root-mean-squared averages, color legend, right). Black dots are speed and feed conditions tested in (n = 3 sampled per dot), while colored contours are interpolated data. Speed and feed conditions that resulted in broken endmills are marked with a red "X". Graphs are arranged in a 3 x 3 matrix representing data for three different plastics (PS, PMMA, COC)), each testing with three different endmill sizes (127, 254, and 508 µm diameters). Resolution in the (B) XY-plane and (C) the vertical z-axis were assessed by comparing the actual size of a fabricated feature to its target "nominal" size (i.e. tolerance or accuracy) (n = 3 samples for all conditions; error bars = standard deviation, represents precision; p = 0.79 via Bartlett test for (B)). (D) SEM micrographs of the features used to characterize the resolution. Red arrows point out the ability to make sharp internal corners via embossing, while rounded fillets form for a pocket made via micromilling.}
\label{figure:MillFig4}
\end{figure}
\FloatBarrier


\subsection{Cell culture}
To test the compatibility of microfabricated devices with cell based experiments, we cultured cells in: (1) micromilled PS devices with two different configurations (one with a flat bottom and one with a milled bottom), (2) a hot-embossed device, and (3) a microtiter plate as a control. PS was selected for its easy machinability and its frequent use in microfluidic cell-based applications \cite{Berthier2012, Berthier2013}. For a variety of mammalian cell types (endothelial, prostate cancer stromal, and bone marrow stromal), we found cell viability was unaffected by the method of device fabrication (Fig. \ref{figure:MillFig5}A), suggesting that microfabricated devices are indeed compatible with cell culture experiments. Notably, cells have previously been observed to respond to surface roughness \cite{Anselme2010, Kieswetter1996, Hatano1999, Deligianni2001, Xu2004, Mustafa2005, Anselme2010, Dowling2011, Gittens2011}. We noticed that cells cultured on the milled surface would occasionally orient along the circular pattern resultant from the milling operation. However, this was not evident in all cases. Furthermore, this issue can easily be avoided by bonding a milled channel to a flat (non-milled) substrate, and culturing the cells on the flat substrate rather than the milled channel. For microscopy imaging, we found that the roughness of micromilled surfaces could impede phase contrast imaging of cells, particularly when cells were cultured directly on the micromilled surfaces (Fig. \ref{figure:MillFig5}B). However, micromilled surface roughness had no observed effect on fluorescent imaging, where image quality was comparable to that obtained with standard microtiter plates (Fig. \ref{figure:MillFig5}B, control).

\begin{figure}[h!] %DONE
\centering
\includegraphics[width=5.5in]{/MillFig5.jpg}
\caption[\textbf{Cell culture and image analysis in milled microchannels.}]{\textbf{Cell culture and image analysis in milled microchannels.} (A) Channels are assembled in three configurations: (1) a milled channel with ports is bonded to a cover layer, (2) a milled port layer is bonded to a milled channel, and (3) an embossed channel with ports is bonded to a cover layer. A microtiter plate is used as a control. Mammalian cell lines were cultured for 48 hours in each configuration, then assayed for cell viability (error bars represent one standard deviation, N = 3). No statistically significant difference was observed between culture methods (p \textgreater 0.14 in all cases, Students T-test). (B) Phase contrast and fluorescent images were taken of HS-5 stromal cells in each channel configuration using 4, 10, and 20x magnifications.}
\label{figure:MillFig5}
\end{figure}


\section{Discussion}
This tutorial demonstrates that micromilling has important utility that enables it to fill several gaps in our current microfabrication repertoire. Micromilling offers excellent versatility across various materials, allowing either the creation of molds for subsequent device production (i.e., hot embossing), or the creation of microchannels and other features directly in devices. Most importantly, micromilling provides design-to-prototype turnaround times on the order of minutes and hours instead of days, weeks, or even months. In this day and age of fast-paced innovation, the ability to prototype a design at this rate can perhaps be the difference between commercializing a product in a year, or being stuck in "development limbo" for much longer \cite{Chin2012}. It is clear that for high-volume production, micromilling cannot compete with the low cost and fast production rates of injection molding and hot embossing. Rather, micromilling excels in the early development stage where frequent design iterations are required in conjunction with the use of conceptualized models, numerical simulations, and other design tools, to converge on an optimal functional design. During this stage, it is both unnecessary and impractical to invest in a series of costly, high-quality molds for high-volume production methods, when the design has not yet been finalized. Another alternative gaining in popularity is to outsource the fabrication to prototyping firms, which promise to work closely and efficiently with their clients during the crucial design phase to generate molds, produce parts, expedite optimization of the design, and ultimately minimize development costs. Outsourcing is available for injection molding, hot embossing, stereolithography, as well as micromilling (e.g., z-microsystems), and thus is feasible and appropriate in many cases. It does, however, require the client to sacrifice some freedom and control on how and when the device is fabricated and delivered. 

Like all fabrication methods, micromilling has advantages and limitations. The main discussion focused on increasing quality of the part in terms of uniformity, accuracy, precision, resolution, and surface roughness. With only an entry-level milling system, we were able to achieve surface roughness of <17 \textmu in (0.42 \textmu m), and similar resolution to hot embossing (Fig. 6D). The surface roughness can be minimized with the techniques described above, but for applications where optical clarity is critical, such as for phase contrast cell microscopy and imaging, the roughness created from the milling process may not be acceptable. These effects are mitigated by bonding transparent PS cover layers, as opposed to imaging directly on a milled surface. Additionally, the roughness of milled surfaces can be reduced afterward with techniques such as solvent vapor polishing, which can in some cases yield surfaces with high optical quality \cite{Ogilvie2010}.

Besides imaging concerns, a common fear is that surface roughness will promote bubble formation, perhaps due to heterogeneities in the surface that perturb the advancing contact line and lead to trapped pockets of air. In our observations over various designs, geometries, and materials, bubble formation has not been an issue in plastic micromilled devices any more than it is an issue for PDMS micromolded devices. If the geometry is not inherently prone to bubble formation, then surface roughness does not promote bubble formation within PMMA, PS, or PC devices. This is likely due to the Wenzel condition for rough surfaces \cite{Wenzel2002RESISTANCEWATER,Ouali2013}, i.e., cos\texttheta\textsubscript{w} = r cos\texttheta\textsubscript{e}, where \texttheta\textsubscript{e} is the Young Law contact angle, \texttheta\textsubscript{w} is the Wenzel contact angle, and r is the ratio between the true contact surface area and projected planar surface area (i.e., r = 1 for smooth surfaces and r \textgreater 1 for rough surfaces). The Wenzel condition states that roughness enhances hydrophobicity if the surface is naturally hydrophobic (\texttheta\textgreater90\textdegree on a smooth surface of the same material). Since aqueous solutions are the most comming priming fluids, and the contact angle of water on PMMA, PS, and PC are all \textless 90\textdegree (reported to be 60\textdegree, 87\textdegree, and 77\textdegree, respectively)\cite{Montez2011} surface roughness will tend to enhance hydrophilicity and promote wetting in devices made from these plastics, and this in turn will reduce the chance of bubble formation. Thus far, our observations have consistently confirmed this prediction.

With regard to production rates, experience combined with strong machining skills can help reduce fabrication time for one device to less than 30 minutes for simple geometries. For more complex designs, fabrication time can range anywhere from 30 minutes to more than an hour. While the operator is free to perform other tasks once the run has been initiated, it is recommended that the operator continue to monitor progress of the run to ensure that it completes without failure.

Fabrication runs do fail occasionally, but most of these failures are caused by common avoidable operational issues. First, endmills will break during a run if there is inadequate chip removal or excessive chip loads (particularly for sub-millimeter diameter endmills). To reduce the frequency of endmill breakage, one can either remove chips efficiently with flood coolant, or use endmills with fewer flutes, which are less likely to trap chips. During the transition from plunge cutting to side milling (i.e., from the Z-axis to milling along the XY-plane) chips often get stuck in the flutes, leading to endmill breakage. This issue can be avoided by using alternative entry methods -- such as a spiral or ramped entry -- wherein the endmill progressively lowers into the material while simultaneously side milling, as opposed to strictly plunging into the material. If endmills are breaking due to excessive chip loads, one can simply reduce feed rate or increase spindle speed. While machining handbooks are excellent resources for finding proper feed rates and speeds for conventional tooling and materials, these parameters may need to be tested for micromilling operations, either through trial and error, or with guidance through calculations based on chip load (i.e., the thickness of the chip that will be removed from each flute). Finally, other possible issues, such as machine vibrations, workpiece vibrations, or offsets in height and setup of the workpiece, should all be carefully examined to help troubleshoot failures. 	

Even with these considerations, we argue that the usefulness of micromilling outweighs its limitations. Indeed, the most significant limitations commonly stated in criticism of micromilling (aside from technical issues) involve its affordability, ease of operation, and suitability for emergent microfluidic applications like cell culture. This review has provided supporting data and evidence to dispel these misconceptions about the challenges of micromilling. First, the capabilities and availability of milling machines are expanding, with a wide selection of equipment to choose from, ranging from high-end advanced systems to affordable systems for the hobbyist. This increases accessibility of the technique to novice machinists and do-it-yourself enthusiasts, which in turn accelerates research, discovery, and innovation. Second, while some mechanical aptitude is necessary to get up and running and to troubleshoot through technical issues, this tutorial guide will hopefully serve as a quick reference to expedite the learning process, and circumvent common pitfalls. Third, cell culture appears to be feasible given our results, although further efforts will be needed to verify specific applications, and further dispel remaining concerns regarding the suitability of milled devices for cell-based studies. Thus, while micromilling has its limitations, the savings it offers in development time and effort during design iterations should alone be worth the investment.

Various other advancements and considerations in the micromilling field are worth noting. First, several manufacturers (e.g., Harvey Tool and Performance Micro Tool) offer endmills with diameters of 0.001 in (25 \textmu m) and smaller. Together with high-end, advanced milling systems, the cutting resolution that can be achieved with such tools will likely reach new limits. While endmills are the workhorse cutters, other more obscure tools such as dragknives can enable cutting 2D contours and small features from thin plastics that would otherwise be difficult to achieve with conventional endmills. For more advanced applications, many 3-axis CNC milling machines can accommodate a fourth axis (i.e., rotation of the workpiece), adding yet another dimension to microfluidic devices that may be impractical with other microfabrication methods. 

As a microfabrication method, micromilling will provide an additional technique that supplements our current repertoire of methods, with specific advantages for handling plastics and other rigid materials. The key advantage of micromilling is its ability to translate designs to prototypes in a matter of minutes and hours, enabling ultra-rapid turnaround times while offering high-quality devices that are suitable for preliminary testing. In addition, complex features can be readily achieved with micromilling, which might otherwise be difficult or impossible to achieve with lithography or embossing. Thus, micromilling can accelerate research and discovery, facilitate innovation, and importantly, contribute to reducing the high costs and long development times that are common to the crucial design phase of technology development. Given these advantages, as well as the available optional accessories and ongoing advancements in technical specifications, micromilling has the potential to play an increased and important role in microfluidics, as well as in other engineering fields. While adding a milling system to your fabrication process is not a simple decision, this tutorial review hopefully offers some useful information to the interested microfluidics researcher.

\section{Acknowledgements}
We thank members of the Microtechnology, Medicine, and Biology lab for their assistance in troubleshooting and general feedback for milling processes. We thank Elizabeth Jin, Brian Johnson, and Ashleigh Theberge for editing and reviewing this manuscript. We thank Tobechukwu Madu and Ryan Lausch for their work with the vapor polishing. We also thank Peter Thomas and Marc Egeland for preliminary contributions and conceptualization of this manuscript. We acknowledge financial support from the University of Wisconsin-Madison Carbone Cancer Center Support Grant NIH P30 CA014520, the Bill and Melinda Gates Foundation, the National Human Genome Research Institute through the Genomic Science Training Program (\#5T32HG002760), and the National Institutes of Health (D. J. G., T. E. d. G., and D. J. B.), Grand Challenges Canada (A. M. D. W.), and the Natural Sciences and Engineering Research Council of Canada (E.W.K.Y).