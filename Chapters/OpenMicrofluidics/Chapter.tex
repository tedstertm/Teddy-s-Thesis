\chapter{Chapter 4: Open Microfluidics}
\label{Chap:OpenMicrofluidics}

Microfluidic systems are fundamentally transforming medicine, diagnostics, and analytical systems, at an academic and industry level. In particular, microfluidics enables a more efficient use of rare samples and reagents(Burns et al., 1998; Ingham et al., 2007; Song, Chen, \& Ismagilov, 2006; Thorsen et al., 2002), observations and measurements at single molecule or cell levels(Hong et al., 2012; Kopp et al., 1998; Song et al., 2006), and engineering platforms of higher relevance to the system being studied(Berthier, Surfus, Verbsky, Huttenlocher, \& Beebe, 2010; Domenech et al., 2009; Huh et al., 2010; Jeon et al., 2015; Meyvantsson \& Beebe, 2008; Moraes et al., 2013; E. K.-H. Sackmann et al., 2014). Despite enabling features, microfluidics come with significant challenges that pose barriers to adoption. Manufacturing enclosed channels at a sub-millimetric scale is a ubiquitous step in traditional microfluidic platforms that increases cost and complexity of fabrication(Guckenberger et al., 2015; E. K. Sackmann, Fulton, \& Beebe, 2014; Shiu et al., 2008). At the industry level, this step often involves highly-guarded trade secrets that slow innovation and development of microfluidic-based products. Usability, reliability, and accessibility are also renown difficulties that limit the field of use and audience for microdevices(Benjamin P Casavant et al., 2013; E. K. Sackmann et al., 2014). Here, we present advances in open microfluidics, a novel approach to designing microfluidic systems that allows the precision of traditional closed microfluidic systems while fundamentally improving affordability, simplicity, and reliability.
The development of simple methods that lower the barriers of adoption for microfluidic systems has increasingly been a focus of research(Berthier et al., 2013; Benjamin P Casavant et al., 2013; Du et al., 2009; Guckenberger et al., 2015; Randall \& Doyle, 2005; Walker et al., 2002; Young et al., 2011). Seminal innovations in microfluidic technologies have proven that a reduction in the complexity of use or fabrication result in an exponential growth in the adoption of these systems(Berthier, Young, \& Beebe, 2012; E. K. Sackmann et al., 2014). The initial widely recognized development in microfabrication that unlocked the field for widespread adoption is soft lithography and the use of silicone polymers for device fabrication(Xia \& Whitesides, 1998). Other developments of importance to medicine and diagnostics that expanded the breadth of applications was the use of thermoplastics through micromilling(Guckenberger et al., 2015; Wilson et al., 2011), hot-embossing(Shiu et al., 2008; Young et al., 2011), or laser cutting(Klank et al., 2002; Yuen et al., 2010). The emergence of high-resolution stereolithography additive printing (SLA, commonly referred to as 3D-printing) is promising to rapidly transform the field of microfluidics. In particular, 3D-printing facilitates the design-to-manufacturing transition and microfluidic assembly, as it enables the production of ready-to-use channels that do not require bonding or other forms of preparation(Au, Huynh, Horowitz, \& Folch, 2016; Bhargava, Thompson, \& Malmstadt, 2014). However, SLA imposes important material limitations (e.g. limited biocompatibility) and uncured resin must be removed from the channels after manufacturing thereby limiting the complexity of the geometries achievable. Simplification of microfluidics technologies has also been demonstrated by using paper fibers as both a path for fluid and a pumping method, showing great promise to develop ultra-low cost diagnostic and detection systems(Martinez, Phillips, \& Whitesides, 2008; Osborn et al., 2010; Park et al., 2013). Paper-based devices are incredibly cost-effective to produce, user-friendly and intuitive to operate, and reliable to use since they don’t rely on pressure sources, seals, and cannot fail due to air bubbles. However fluid handling in paper is limited, in particular the ability to exchange and wash fluids, and material interaction and adsorption challenges are increased due to the high surface-to-volume ratios. The common theme of microfluidic development is the reduction of manufacturing and user-operation cost and complexity. There remains, however, a tradeoff between control over the fluid and operational simplicity.
Open microfluidics describes the flow of fluids in channels that have one or more open faces; the simplest being a rectangular channel with no ceiling(Berthier, Theberge, et al., 2012; B.P. Casavant et al., 2013). Open microfluidics allows the control over fluids on-par with traditional closed-fluidics while providing manufacturing simplicity and reliability of use on par with paper-microfluidics. In open microfluidics, fluid flow is driven directed by capillary force provided by the solid parts of the cross section of the channel allowing the fluid to span over the open sections of the cross section. We compile here the initial forays of open microfluidics and expand on them to demonstrate the potential of open microfluidics to further democratize the prototyping and manufacturing of microfluidic-based systems that are reliable and low-cost. We show that open microfluidics is a ubiquitous technology that enables control over fluid handling, access to the sample throughout the device operation, simple manufacturing and prototyping (compatible with injection molding and 3D printing), and the potential to create non-planar reconfigurable systems. We present a suite of devices that are driven by open microfluidics and designed to overcome many limitations that current microfluidic platforms have to adoption research communities at large. Finally, we show that reconfigurable systems can be created utilizing the example of Lego-like blocks that enable the creation of a bread-board microfluidic system that can be modified during the operation of the device [cite Jason?].
