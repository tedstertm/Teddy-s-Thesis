\chapter{Introduction}
\label{Chap:Introduction}

\textit{In vitro} techniques to model mammalian biology have long depended on the use of 2D cell culture in polystyrene dishes, while rat and mouse \textit{in vivo} models have been used to study whole organisms. These two tools, while at opposite ends of the spectrum of model complexity have provided us with much of what we know about human biology and have led to the development of countless therapies to treat a spectrum of human malignancies. Translating discoveries into useful medicine is incredibly difficult and new compounds entering FDA phase 1 clinical trials have a 10-15\% likelihood of approval \cite{Hay2014}. Many of these compounds enter clinical trials after \textit{in vitro} or \textit{in vivo} models show initial success. The low success rates show that these models have inherent artifacts resulting in the failure to adequately reproduce appropriate human biological and physiological responses to many compounds and may call into question the relevance of these models to other biological phenomena they were used to study. Simple 2D models lack physiological context to be truly relevant, while many discoveries made in rodents do not transfer to humans due to differences in physiology. Issues with classical biological models have have not gone unnoticed and there have been great efforts to develop engineered \textit{in vitro} models that can replicate human tissues, organs, and even organ systems. The rationalization is that increased but controllable complexity will lead to increased relevance of the models and biological insights gained from studying them can be better applied to human medicine. I describe how engineered \textit{in vitro} microscale models can be designed, fabricated, and used to investigate key components such as cell-cell, cell-matrix, and cell-drug interactions in the complex space of the tumor microenvironment.

\section{Developing an appropriate \textit{in vitro} model}

\textit{In vitro} models can range in complexity to a simple culture of cells in a polystyrene flask all the way to body-on-a-chip constructs that link organ and organoid cultures to form functional organ systems \cite{Esch2011a}. One of the most important considerations for designing an \textit{in vitro} model is choosing the appropriate complexity to work with. Complexity generally needs to align with the questions being asked of the model, for example, a researcher measuring motility of a cell line can usually rely on a simple scratch assay to get appropriate results \cite{liang2007vitro}. In this scenario, a more complex model, such as engineered organoid will likely obfuscate the results as more factors are integrated into the system and will the goal of the study (fibroblast motility) significantly more difficult to observe and measure. Conversely using a scratch assay to study cell motility in a much more complex scenario, such as cell extravasation across a blood vessel in tumor metastasis would be inadequate as it lacks the relevant context to the system being modeled. Unfortunately the appropriate level of complexity for the later scenario isn't entirely clear, which components will need to be included in the model to get an accurate biological representation are debatable \cite{Bouet2015, davies2015capturing, stock2016capturing, Munch2000} and in some cases complexity that is known to give relevance to a model may need to be sacrificed for throughput for applications like drug screening \cite{Hickman2014Three-dimensionalVivo.}. However complicated the question of how complex to make a model, a solid understanding of the \textit{in vivo} microenvironment of the system being replicated and knowledge of the tools available to achieve the level of complexity desired are essential designing viable models. 

\subsection{The \textit{in vivo} microenvironment}
Microenvironment is a term used to describe all the cells, the structure, and the signals present in a given tissue niche. The types of cells and their interactions usually define the function of the tissue or microenvironment. For example the pancreatic islet microenvironment contains several cell types, most notably \textalpha \ and \textbeta \ cells responsible for producing hormones to regulate blood glucose and digestion \cite{Cabrera2006TheFunction.}. Cells in the microenvironment are structurally supported and physically arranged within the extracellular matrix (ECM). The ECM provides for cell adhesion, regulates stiffness, and regulates cell behavior through interactions with cell-surface proteins \cite{Stamenkovic2003, Meredith1993TheFactor.}. The signaling component of the microenvironment consists of all proteins, peptides, and molecules that cells can interact with and that will modify cell behavior. These signals can be in the form of cell-cell interactions through CAMs, soluble peptides in the ECM, small molecules, or chemical stresses like pH and oxygen \cite{Kim2011b}. These components are intricately interconnected and both drive and define the microenvironment.

The microenvironment is different between tissues, and between different states of tissue (healthy/diseased) so components will be unique to each microenvironment being modeled. A regulated microenvironment is important for maintaining homeostasis \cite{Nie2013Microenvironment-dependentRepair}, while the loss of regulation is linked to a number of diseases including cancer \cite{Malizia1985TheLymphocytes., Yip2011, medema2011microenvironmental, Charbord1996EarlyMicroenvironment., Martin1995, Nie2013Microenvironment-dependentRepair}. The tumor microenvironment is perhaps most studied, it has been observed in many types of cancer that the microenvironment and the cancer cells themselves can act on each other to result in advancement of the disease and a more malignant phenotype \cite{Meads2008, Jun2006, Shain2001CellMDR., Seo2015}.

\subsection{The \textit{in vitro} microenvironment}
There are many tools and techniques available to replicate components of the \invivo\ microenvironment \invitro. Natural and synthetic structural materials are available to recreate various components of the ECM. In this discussion natural materials are those native to mammalian physiology including proteins, peptides, and other polymers like glycosaminoglycans whereas synthetic refers to not only synthesized polymers but other non-native polymers like aglinate and agarose. Natural materials can be available as single components like collagen I or as mixtures that are collected from cell secretions that mimic specific microenvironmental ECM like Matrigel which is like a basement membrane. The biggest advantage of using natural ECM materials is that they are what is found \invivo\ cells already recognize binding sites on natural ECM and have the machinery to remodel and reconstruct them \cite{Mano2007NaturalTrends.}. ECM mixtures like Matrigel are well defined but can be variable from batch between batches and have growth factors from production embedded these two factors can lead to unexpected biological changes in culture with possible impact on results \cite{Hughes2010Matrigel:Culture.}. The main advantage for use of synthetic ECM in the construction of \invitro\ microenvironments is control. Synthetic materials allow for precise modulation a wide range of variables that impact cell behavior including matrix stiffness \cite{Peyton2008TheSystem}, cell-matrix engagement \cite{Hsiong2008Integrin-AdhesionMatrices}, surface modification \cite{Jongpaiboonkit2009GrowthMicrospheres.}, soluble factor release \cite{Belair2016DifferentialMicrospheres}, matrix degradation over time \cite{Gunatillake2003BiodegradableEngineering.} or through cell-induced cleavage \cite{Tibbitt2009HydrogelsCulture}. While tunable, and consistent, and ever expanding functionality at this point synthetic materials cannot account for the complexity and subtlety that natural ECM materials have. There is an incredibly large variety in materials that can be used to construct ECM for \invitro\ cultures. The materials chosen to construct the ECM for the model being developed is called a scaffold, the scaffold is used to culture the cells in the model. 

The choice of cells in an \invitro\ model is a little more straight forward than the choice of ECM materials. The biggest concern for choosing cells is how much complexity they will add to the model. Measuring interactions between cells and each other or the microenvironment becomes significantly more complicated as cell types are added. The addition of more cell types and the order they are added can have a significant impact cell structure, cell behavior and matrix reorganization. The order of addition of cells to the model would be critical if investigating how disease states change healthy microenvironments to diseased ones. 

An artifact of \invitro\ culture that also needs consideration is mechanical stress on cells. 

Alignment, cilia development. 

\section{Microfluidics for cell culture}

There are many tools available for implementing the \invitro\ tools described here. Bioreactors for tissue engineering applications that go beyond just a stirred flask have been developed to provide uniform media flow through scaffolds, fluid shear across surfaces, and mechanical strain to tissue constructs to allow the \invitro\ culture and development of many tissues \cite{Martin2004}. While many bioreactors have been designed for larger tissue constructs for tissue engineering applications, microfluidic bioreactors have been developed capable of the same functions as their macroscale companions and much more\cite{}. 

Microfluidics we well suited for development of \invitro\ culture platforms. Microfluidic and microscale devices require very small volumes of fluid for operation, which can be very enabling. Low volumes result reduced need for media and reagents saving costs \cite{}. Low volumes are amenable to high throughput processing applications where a limited sample, for example a primary patient sample, can be exposed to many experiment conditions and increase the amount of data that can be retrieved from a patient \cite{}. 

Microscale phenomena give microfluidics some unique advantages as water behaves in unintuitive, but predictable ways \cite{}. At small volumes the Reynold's number of the system become low (viscous forces outweigh inertial forces) and the effects of gravity are diminished \cite{}. Macroscale phenomena like turbulence is absent, fluid flow at the microscale is laminar, flowing in sheets, mixing of fluids is largely dependent on diffusion \cite{}. Microscale phenomena can be overcome and fluid can be pushed to mix using tricks like increasing local chaos \cite{}, or they can be embraced to to enable new \invitro biology. Diffusion gradients, passive and active can be established to serve as chemoattractants \cite{}\ for migration experiments or to simulate aspects of development \cite{}.

Syringe and peristaltic pumps, the traditional tools used to drive microfluidic platforms are great tools for developing biomimetic \invitro\ culture models. Pumps allow for the precise control fluid flow in microchannels, allowing researchers to achieve physiologically relevant mechanical shear across cells to model many biological phenomena \cite{}. 

\subsection{Materials for Microfluidics}
As argued by Berthier \textit{et al.}\ a significant obstacle to adoption of engineered microfluidic devices and culture platforms by biologists is culture material. Biologists have used and still use polystyrene (PS) flasks, dishes, and plates, while polydimethylsiloxane (PDMS) has been the primary material for prototyping microfluidic devices \cite{Berthier2012} . While distrust of new and known materials is part of the barrier for adoption, there many reasons why PDMS is not ideal for biological work, however, polystyrene may not be an ideal material either.

PDMS has been a choice material for engineers to prototype microchannels and microfluidic devices for biological uses \cite{Sia2003a}. The material itself is easy to work with, just mix the polymer with a crosslinker and pout onto a mold, crosslinking will happen spontaneously or can be sped up with the addition of heat. It can fill very small features and be easily cast from a mold while achieving little wear due to relatively low stiffness and high compliance. PDMS can be easily and irreversibly bound to glass making bonding simple and repeatable. 

While an excellent material for prototyping microfluidic devices, PDMS has several features that make it unsuitable some biological applications. One the most concerning and apparent issues is it's hydrophobicity. Native PDMS has a contact angle of over 110\textdegree\, which can be reduced to less than 45\textdegree\ following plasma treatment \cite{Mata2005CharacterizationMicro/Nanosystems}. Plasma treatment is temporary as PDMS will being to undergo hydrophobic recovery immediately following treatment and will return to its native state within hours if no extra precautions are taken \cite{Eddington2006ThermalPolydimethylsiloxane}. A highly hydrophobic material will adsorb and sequester small hydrophobic molecules such as steroids or other small molecules lowering the concentration in solution. PDMS, unlike hydrophobic thermoplastics or thermoplastic elastomers allows diffusion of small molecules into the bulk of the material acting as a constant sink whereas thermoplastics will become saturated at the surface \cite{Toepke2006, Regehr2009,Borysiak2013b}. Cells cultured in PDMS microchannels have been observed to sequester uncrosslinked PDMS oligamers within their cytoplasm, while this hasn't been observed to affect viability the biological implications of this observation is unknown \cite{Regehr2009}.

PS and PDMS are just a small sampling of the materials available to fabricate microfluidic devices with. Cyclic olefin copolymer is block copolymer that has some attractive properties for cell culture including high optical clarity and low autofluorescence compared to PS and is cell-compatible\cite{Young2012, VanMidwoud2012}. Thermoplastic elastomers are polymers that have a relatively large temperature difference between their glass transition state and melting point where they behave as a rubber. Styrene-based thermoplastic elastomers like styrene-ethylene/butylene-styrene (SEBS) have several properties making them attractive materials for tissue culture and microdevice fabrication. SEBS can be made soft and compliant like PDMS, is amenable to the same fabrication techniques as most thermoplastics, and while hydrophobic does allow for the diffusion of small molecules into the bulk of the material \cite{Borysiak2013b, Borysiak2013,Roy2011,Guillemette2011,Guillemette2009,Li2013a}.

One of the more interesting materials for microdevice fabrication is paper. Paper can be patterned with wax printing to create hydrophobic and hydrophilic regions to guide fluid wicking through the substrate \cite{Carrilho2009}. While this material has been traditionally used to develop low-cost lateral flow assays it has more recently been used as a substrate for 3D cell culture \cite{Derda2009}. Paper has poor optical qualities, however, this can be somewhat overcome with design. An invasion assay was developed where many sheets of thin paper were stacked together, cells were seeded into the stack, then separated for imaging \cite{Derda2011}.

All systems and materials have specific and unique artifacts associated with them and PS is not an exception. The stiffness of polystyrene is on same order of magnitude as that of bone, stiffer than PDMS and much stiffer than that tissue where most cell lines were derived from \cite{Saha2008SubstrateBehavior}. High stiffness results in reorganization of the cytoskeleton \cite{Discher2005TissueSubstrate.}, and is associated with the promotion of malignant phenotypes in several cancers \cite{Paszek2005TensionalPhenotype, Seo2015, Verbridge2010Tissue-engineeredAngiogenesis.}. Several cell types do not culture well directly on PS, maintenance of stem cells require direct culture on fibroblasts or gel \cite{Okita2007, Xu2001}. Additionally cells have been shown to behave considerably differently when cultured in in a 3D scaffold instead of on the surface of PS \cite{Cukierman2001, Pampaloni2007TheTissue, Sung2013}. In spite of the known and unknown artifacts PS attributing to tissue culture, PS has produced a lot of data, practically all tissue culture protocols have been developed for it, and it's currently the best and easiest way to maintain and expand cell cultures.

\subsection{Device fabrication}
Device fabrication is discussed in depth in chapter \ref{Chap:Micromilling}, so a brief description of the techniques available will be discussed here. PDMS casting is a low throughput method for producing microdevices and requires fairly specialized equipment. Since it has limited scalability and plasma treatment of PDMS devices is very short term resulting in difficult filling of channels, PDMS devices used in biology will likely need to be sourced from the labs that designed them or fabricated by a handful of university foundries equipped to produce custom devices. Thermoplastics and thermoplastic elastomers like PS, cyclic olefin copolymer, poly(methyl methacrylate), and many more are amenable to a wide variety of fabrication techniques from milling \cite{Guckenberger2015b}, 3d printing \cite{Au2014}, solvent casting \cite{Borysiak2013b}, and hot embossing \cite{Becker2000} at the prototyping scale to injection molding \cite{Piotter1997} for high volume fabrication. 

\subsection{Microfluidic platform operation}
Microfluidic and microscale techniques offer many significant advantages for the development of \invitro\ tissue culture platforms but aren't widely used by the biology community. 


\section{Multiple myeloma}
MM is kind of difficult to study. It's crazy complex


\subsection{MM microenvironment}
All the different variables


\subsection{MM engineering}

Talk about chorom and edmond's work being a place where they had good collaboration 

Developing tools to increase the relevance of the model without increasing the complexity or barrier to entry for it. 
