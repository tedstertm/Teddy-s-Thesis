\chapter*{Introduction} % Write in your own chapter title
\label{Introduction}
\addcontentsline{toc}{chapter}{Introduction}
Although physical principles in general do not differ between macro- and micro-scale systems, the relative importance of basic physical forces and processes such as gravity, surface-tension, viscosity, convection, and diffusion does. This change creates an opportunity to develop fundamentally new solutions to old problems and enables new avenues of discovery. My studies have focused on understanding microscale forces and processes and how they can be leveraged for the study of cellular interactions such as adhesion and soluble factor signaling. Figure \ref{Chap:Introduction:fig:relationships} helps to put each of these studies into context, categorizing them at a high-level into three categories: examinations of the microscale phenomena and their implications, characterizations of components that leverage microscale phenomena, and development of biological assays that leverages those components and phenomena. Although each study has a different focus, they each support and feed back on one another with the ultimate goal of providing tools to functionally examine common processes or recurring hypotheses of cellular interactions.
\begin{figure}[!ht]
\centering
\includegraphics[width=5in]{IntroDiagram.pdf}
\caption{\textbf{Project diagram}.}
\label{Chap:Introduction:fig:relationships}
\end{figure}

The following chapters and appendices provide the details of these studies, beginning with summaries of work characterizing the basic processes and concepts of performing cell-based assays at the micro-scale including evaporation, diffusion, fluid-flow in microchannels, and reduced volumes. Discussion then turns towards individual components that leverage microscale phenomena to provide very specific functionality in a flexible and modular way. The final chapters describe a new technique for studying cell adhesion and its use to characterize tumor-cell adhesion to endothelia as well as bone marrow mesenchymal stem-cell adhesion to a new \invitro\ model of cardiac extracellular matrix.
