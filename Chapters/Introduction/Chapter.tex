\chapter{Chapter 1: Introduction}\label{Chap:Introduction}


\section{Preface}
\textit{In vitro} techniques to model mammalian biology have long depended on the use of 2D cell culture in polystyrene dishes, while rat and mouse \textit{in vivo} models have been used to study whole organisms. These two tools, while at opposite ends of the spectrum of model complexity have provided us with much of what we know about human biology and have led to the development of countless therapies to treat a spectrum of human malignancies. Translating discoveries into useful medicine is incredibly difficult and new compounds entering FDA phase 1 clinical trials have a 10-15\% likelihood of approval (Hay, 2014). Many of these compounds enter clinical trials after \textit{in vitro} or \textit{in vivo} models show initial success. The low success rates show that these models have inherent artifacts resulting in the failure to adequately reproduce appropriate human biological and physiological responses to many compounds and may call into question the relevance of these models to other biological phenomena they were used to study.  



\section{Microfluidics for Tissue culture}






