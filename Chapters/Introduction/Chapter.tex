\chapter{Introduction}
\label{Chap:Introduction}

\textit{In vitro} techniques to model mammalian biology have long depended on the use of 2D cell culture in polystyrene dishes, while rat and mouse \textit{in vivo} models have been used to study whole organisms. These two tools, while at opposite ends of the spectrum of model complexity have provided us with much of what we know about human biology and have led to the development of countless therapies to treat a spectrum of human malignancies. Translating discoveries into useful medicine is incredibly difficult and new compounds entering FDA phase 1 clinical trials have a 10-15\% likelihood of approval \cite{Hay2014}. Many of these compounds enter clinical trials after \textit{in vitro} or \textit{in vivo} models show initial success. The low success rates show that these models have inherent artifacts resulting in the failure to adequately reproduce appropriate human biological and physiological responses to many compounds and may call into question the relevance of these models to other biological phenomena they were used to study. Simple 2D models lack physiological context to be truly relevant, while many discoveries made in rodents do not transfer to humans due to differences in physiology. Issues with classical biological models have have not gone unnoticed and there have been great efforts to develop engineered \textit{in vitro} models that can replicate human tissues, organs, and even organ systems. The rationalization is that increased but controllable complexity will lead to increased relevance of the models and biological insights gained from studying them can be better applied to human medicine. I describe how engineered \textit{in vitro} microscale models can be designed, fabricated, and used to investigate key components such as cell-cell, cell-matrix, and cell-drug interactions in the complex space of the tumor microenvironment, but still face some large hurtles to wide-spread adoption and acceptance by the biology community.

\section{Developing an appropriate \textit{in vitro} model}

\textit{In vitro} models can range in complexity to a simple culture of cells in a polystyrene flask all the way to body-on-a-chip constructs that link organ and organoid cultures to form functional organ systems \cite{Esch2011a}. One of the most important considerations for designing an \textit{in vitro} model is choosing the appropriate complexity to work with. Complexity generally needs to align with the questions being asked of the model, for example, a researcher measuring motility of a cell line can usually rely on a simple scratch assay to get appropriate results \cite{liang2007vitro}. In this scenario, a more complex model, such as engineered organoid will likely obfuscate the results as more factors are integrated into the system and will the goal of the study (fibroblast motility) significantly more difficult to observe and measure. Conversely using a scratch assay to study cell motility in a much more complex scenario, such as cell extravasation across a blood vessel in tumor metastasis would be inadequate as it lacks the relevant context to the system being modeled. Unfortunately the appropriate level of complexity for the later scenario isn't entirely clear, which components will need to be included in the model to get an accurate biological representation are debatable \cite{Bouet2015, davies2015capturing, stock2016capturing, Munch2000} and in some cases complexity that is known to give relevance to a model may need to be sacrificed for throughput for applications like drug screening \cite{Hickman2014Three-dimensionalVivo.}. However complicated the question of how complex to make a model, a solid understanding of the \textit{in vivo} microenvironment of the system being replicated and knowledge of the tools available to achieve the level of complexity desired are essential designing viable models. 

\subsection{The \textit{in vivo} microenvironment}
Microenvironment is a term used to describe all the cells, the structure, and the signals present in a given tissue niche. The types of cells and their interactions usually define the function of the tissue or microenvironment. For example the pancreatic islet microenvironment contains several cell types, most notably \textalpha \ and \textbeta \ cells responsible for producing hormones to regulate blood glucose and digestion \cite{Cabrera2006TheFunction.}. Cells in the microenvironment are structurally supported and physically arranged within the extracellular matrix (ECM). The ECM provides for cell adhesion, regulates stiffness, and regulates cell behavior through interactions with cell-surface proteins \cite{Stamenkovic2003, Meredith1993TheFactor.}. The signaling component of the microenvironment consists of all proteins, peptides, molecules, and mechanical signals that cells can interact with and that will modify cell behavior. These signals can be in the form of cell-cell interactions through CAMs, soluble peptides in the ECM, small molecules, or chemical stresses like pH and oxygen, shear flow in blood vessels or even mechanical compression from impact \cite{Kim2011b}. These components are intricately interconnected and both drive and define the microenvironment.

The microenvironment is different between tissues, and between different states of tissue (healthy/diseased) so components will be unique to each microenvironment being modeled. A regulated microenvironment is important for maintaining homeostasis \cite{Nie2013Microenvironment-dependentRepair}, while the loss of regulation is linked to a number of diseases including cancer \cite{Malizia1985TheLymphocytes., Yip2011, medema2011microenvironmental, Charbord1996EarlyMicroenvironment., Martin1995, Nie2013Microenvironment-dependentRepair}. The tumor microenvironment is perhaps most studied, it has been observed in many types of cancer that the microenvironment and the cancer cells themselves can act on each other to result in advancement of the disease and a more malignant phenotype \cite{Meads2008, Jun2006, Shain2001CellMDR., Seo2015}.

\subsection{The \textit{in vitro} microenvironment}
There are many tools and techniques available to replicate components of the \invivo\ microenvironment \invitro. Natural and synthetic structural materials are available to recreate various components of the ECM. In this discussion natural materials are those native to mammalian physiology including proteins, peptides, and other polymers like glycosaminoglycans whereas synthetic refers to not only synthesized polymers but other non-native polymers like aglinate and agarose. Natural materials can be available as single components like collagen I or as mixtures that are collected from cell secretions that mimic specific microenvironmental ECM like Matrigel which is like a basement membrane. The biggest advantage of using natural ECM materials is that they are what is found \invivo\ cells already recognize binding sites on natural ECM and have the machinery to remodel and reconstruct them \cite{Mano2007}. ECM mixtures like Matrigel are well defined but can be variable from batch between batches and have growth factors from production embedded these two factors can lead to unexpected biological changes in culture with possible impact on results \cite{Hughes2010Matrigel:Culture.}. The main advantage for use of synthetic ECM in the construction of \invitro\ microenvironments is control. Synthetic materials allow for precise modulation a wide range of variables that impact cell behavior including matrix stiffness \cite{Peyton2008TheSystem}, cell-matrix engagement \cite{Hsiong2008Integrin-AdhesionMatrices}, surface modification \cite{Jongpaiboonkit2009GrowthMicrospheres.}, soluble factor release \cite{Belair2016DifferentialMicrospheres}, matrix degradation over time \cite{Gunatillake2003BiodegradableEngineering.} or through cell-induced cleavage \cite{Tibbitt2009HydrogelsCulture}. While tunable, and consistent, and ever expanding functionality at this point synthetic materials cannot account for the complexity and subtlety that natural ECM materials have. There is an incredibly large variety in materials that can be used to construct ECM for \invitro\ cultures. The materials chosen to construct the ECM for the model being developed is called a scaffold, the scaffold is used to culture the cells in the model. 

The choice of cells in an \invitro\ model is a little more straight forward than the choice of ECM materials. The biggest concern for choosing cells is how much complexity they will add to the model. Measuring interactions between cells and each other or the microenvironment becomes significantly more complicated as cell types are added. The addition of more cell types and the order they are added can have a significant impact cell structure, cell behavior and matrix reorganization. The order of addition of cells to the model would be critical if investigating how disease states change healthy microenvironments to diseased ones. 

Controlling signalling in an \invitro\ model is essential to studying microenvironments. Small changes in signaling can result in healthy tissue becoming diseased \cite{Ziyad2011} and understanding these processes can lead to a better understanding of disease progression and lead to potential therapies for treatment. Much signaling will be out of the direct control of the researcher as it arises from cell-cell and cell-matrix interactions, however external signals such as drugs and naturally occurring molecules can be added to reach a desired phenotypic outcome or induce perturbations in the model. The use of VEGF in \invitro\ angiogensis models is a well studied example of this. Beyond chemical signalling, mechanical signalling can be controlled to alter the behavior of the \invitro\ microenvironment. Mechanical signals can come from the scaffold itself in terms of stiffness and choice of material. Fibronectin is a molecule that changes conformation and exposes additional binding site for cells under mechanical tension. Mechanical signals can come in the form of shear stress over a layer of cells and is sometimes required for viability \cite{White2007, Dimmeler1996, Traub1998} or by physical deformation of the parts of the entire culture, which has been shown to be important for muscle development \cite{Leung1976, Humphrey2014, Back2013}. Both mechanical shear and deformation have been used in creating biomimetic models of lung and gut \cite{Kim2013a,  Huh2010a}. 

\section{Microfluidics for cell culture}

\subsection{Microfluidic tools and concepts}
There are many tools available for implementing the \invitro\ tools described here. Bioreactors for tissue engineering applications that go beyond just a stirred flask have been developed to provide uniform media flow through scaffolds, fluid shear across surfaces, and mechanical strain to tissue constructs to allow the \invitro\ culture and development of many tissues \cite{Martin2004}. While many bioreactors have been designed for larger tissue constructs for tissue engineering applications, microfluidic bioreactors have been developed capable of the same functions as their macroscale companions and much more\cite{Andersson2004, Khademhosseini2006, Choi2007a}. 

Microfluidics we well suited for development of \invitro\ culture platforms. Microfluidic and microscale devices require very small volumes of fluid for operation, which can be very enabling. Low volumes result reduced need for media and reagents saving costs \cite{Ohno2008a, Situma2006}. Low volumes are amenable to high throughput processing applications where a limited sample, for example a primary patient sample, can be exposed to many experiment conditions and increase the amount of data that can be retrieved from a patient \cite{Young2012}. 

Microscale phenomena give microfluidics some unique advantages as water behaves in unintuitive, but predictable ways. At small volumes the Reynold's number of the system become low (viscous forces outweigh inertial forces) \cite{Beebe2002a}. Macroscale phenomena like turbulence is absent, fluid flow at the microscale is laminar, flowing in sheets, mixing of fluids is largely dependent on diffusion \cite{purcell1977life}. Microscale phenomena can be overcome and fluid can be pushed to mix using tricks like increasing local chaos \cite{DeMello2006, Stroock2004}, or they can be embraced to to enable new \invitro biology. Diffusion gradients, passive and active can be established to serve as chemoattractants \cite{Keenan2008} for migration experiments or to simulate aspects of development \cite{Berthier2014a}. At small volumes high surface tension at the interface between water and air becomes apparent and has a greater influence on gravity on fluid behavior, resulting in fluid changing shape to minimize its free energy by reducing its contact area with air and increasing its contact area with hydrophilic surfaces present. This phenomenon can observed in the spontaneous capillary filling of small spaces against gravity. It has been frequently exploited for passive or spontaneous fluid movement in microfluidic devices.

\subsection{Materials for Microfluidics}
As argued by Berthier \textit{et al.}\ a significant obstacle to adoption of engineered microfluidic devices and culture platforms by biologists is culture material. Biologists have used and still use polystyrene (PS) flasks, dishes, and plates, while polydimethylsiloxane (PDMS) has been the primary material for prototyping microfluidic devices \cite{Berthier2012} . While distrust of new and known materials is part of the barrier for adoption, there many reasons why PDMS is not ideal for biological work, however, polystyrene may not be an ideal material either.

PDMS has been a choice material for engineers to prototype microchannels and microfluidic devices for biological uses \cite{Sia2003a}. The material itself is easy to work with, just mix the polymer with a crosslinker and pout onto a mold, crosslinking will happen spontaneously or can be sped up with the addition of heat. It can fill very small features and be easily cast from a mold while achieving little wear due to relatively low stiffness and high compliance. PDMS can be easily and irreversibly bound to glass making bonding simple and repeatable. 

While an excellent material for prototyping microfluidic devices, PDMS has several features that make it unsuitable some biological applications. One the most concerning and apparent issues is it's hydrophobicity. Native PDMS has a contact angle of over 110\textdegree\, which can be reduced to less than 45\textdegree\ following plasma treatment \cite{Mata2005CharacterizationMicro/Nanosystems}. Plasma treatment is temporary as PDMS will being to undergo hydrophobic recovery immediately following treatment and will return to its native state within hours if no extra precautions are taken \cite{Eddington2006ThermalPolydimethylsiloxane}. A highly hydrophobic material will adsorb and sequester small hydrophobic molecules such as steroids or other small molecules lowering the concentration in solution. PDMS, unlike hydrophobic thermoplastics or thermoplastic elastomers allows diffusion of small molecules into the bulk of the material acting as a constant sink whereas thermoplastics will become saturated at the surface \cite{Toepke2006, Regehr2009,Borysiak2013b}. Cells cultured in PDMS microchannels have been observed to sequester uncrosslinked PDMS oligamers within their cytoplasm, while this hasn't been observed to affect viability the biological implications of this observation is unknown \cite{Regehr2009}.

PS and PDMS are just a small sampling of the materials available to fabricate microfluidic devices with. Cyclic olefin copolymer is block copolymer that has some attractive properties for cell culture including high optical clarity and low autofluorescence compared to PS and is cell-compatible\cite{Young2012, VanMidwoud2012}. Thermoplastic elastomers are polymers that have a relatively large temperature difference between their glass transition state and melting point where they behave as a rubber. Styrene-based thermoplastic elastomers like styrene-ethylene/butylene-styrene (SEBS) have several properties making them attractive materials for tissue culture and microdevice fabrication. SEBS can be made soft and compliant like PDMS, is amenable to the same fabrication techniques as most thermoplastics, and while hydrophobic does allow for the diffusion of small molecules into the bulk of the material \cite{Borysiak2013b, Borysiak2013,Roy2011,Guillemette2011,Guillemette2009,Li2013a}.

One of the more interesting materials for microdevice fabrication is paper. Paper can be patterned with wax printing to create hydrophobic and hydrophilic regions to guide fluid wicking through the substrate \cite{Carrilho2009}. While this material has been traditionally used to develop low-cost lateral flow assays it has more recently been used as a substrate for 3D cell culture \cite{Derda2009}. Paper has poor optical qualities, however, this can be somewhat overcome with design. An invasion assay was developed where many sheets of thin paper were stacked together, cells were seeded into the stack, then separated for imaging \cite{Derda2011}.

All systems and materials have specific and unique artifacts associated with them and PS is not an exception. The stiffness of polystyrene is on same order of magnitude as that of bone, stiffer than PDMS and much stiffer than that tissue where most cell lines were derived from \cite{Saha2008SubstrateBehavior}. High stiffness results in reorganization of the cytoskeleton \cite{Discher2005TissueSubstrate.}, and is associated with the promotion of malignant phenotypes in several cancers \cite{Paszek2005TensionalPhenotype, Seo2015, Verbridge2010}. Several cell types do not culture well directly on PS, maintenance of stem cells require direct culture on fibroblasts or gel \cite{Okita2007, Xu2001}. Additionally cells have been shown to behave considerably differently when cultured in in a 3D scaffold instead of on the surface of PS \cite{Cukierman2001, Pampaloni2007TheTissue, Sung2013}. In spite of the known and unknown artifacts PS attributing to tissue culture, PS has produced a lot of data, practically all tissue culture protocols have been developed for it, and it's currently the best and easiest way to maintain and expand cell cultures.

\subsection{Device fabrication}
Device fabrication is discussed in depth in chapter \ref{Chap:Micromilling}, so a brief description of the techniques available will be discussed here. PDMS casting is a low throughput method for producing microdevices and requires fairly specialized equipment. Since it has limited scalability and plasma treatment of PDMS devices is very short term resulting in difficult filling of channels, PDMS devices used in biology will likely need to be sourced from the labs that designed them or fabricated by a handful of university foundries equipped to produce custom devices. Thermoplastics and thermoplastic elastomers like PS, cyclic olefin copolymer, poly(methyl methacrylate), and many more are amenable to a wide variety of fabrication techniques from milling \cite{Guckenberger2015b}, 3d printing \cite{Au2014}, solvent casting \cite{Borysiak2013b}, and hot embossing \cite{Becker2000} at the prototyping scale to injection molding \cite{Piotter1997} for high volume fabrication. 

\subsection{Microfluidic platform use}
Microfluidic and microscale techniques offer many significant advantages for the development of \invitro\ tissue culture platforms and many platforms have been around for a fairly long time, however, microfluidics has not seen widespread adoption. There has been much discussion on why microfluidic technologies have been experiencing difficulties transferring from engineering lab to the biology lab. One of the issues often brought up is the usability and it more often than not revolves around external pump. The pro-pump camp argues that pumps give you precise control over all fluidic aspects of the system and the problem is that with so many labs developing unique systems there's no standard way to interface pump and chip, the development of a plug-and-play interface standard would go a long way to solving the adoption issue \cite{Scott2013, VanHeeren2012}. The anti-pump side argues that the best way to facilitate adoption is by developing technologies that can be operated tools biologists are already familiar with, like pipettes.  Fluid in these systems can be operated passively through gravity or passive pumping, which gives the operator significantly less control over the fluid parameters, but doesn't require extra equipment investment or extensive training \cite{Walker2002, Lee2015}. Both sides make valid points but there has been a lot of work done to standardize how pumps interface with devices (ThinXXS, microfluidic ChipShop, Elveflow, Dolomite), and passively driven microfluidic techniques have been available for well over a decade \cite{Weigl2000}.

Choice of materials for making microfluidic devices is important, and few people would argue against a simple and usable interface with a microfluidic system, but the issue of specificity is rarely raised as a barrier to adoption of microfluidic technologies in biology settings. Microfluidic technology in biology labs have been increasing, but mostly for biochemistry applications, where the fluidics of the system are sitting in a black box. Operation is usually simple, load reagents, run a program on a computer attached to the device, and collect data. These tools, while usually specific in function (DNA amplification (Fluidigm), protein quantification (Luminex), sequencing prep (10x Genomics), etc.) are amenable to many biological systems, can produce volumes of data, their expense could be justified as part of an equipment core. Microfluidic \invitro\ models are usually very specific, designed with the flow and geometry to investigate a particular system. The need to reconfigure not only culture protocol and materials, but the device geometry is not amenable to widespread adoption. Millipore sells a microfluidic tissue culture perfusion system that in a well-plate format. The majority of the real estate on the system is occupied by wells and channels to accommodate all the fluidic manipulations users may want to do on the system, leaving only 4 2.8mm wide culture wells. To make microfluidics viable in biology labs, there needs to be platforms that generic enough to make manufacturing worth the cost, while minimizing loss of function.


\section{Multiple myeloma}
Multiple myeloma (MM) is often described as a cancer of plasma cells. It is the second most common hematological malignancy and accounts for approximately 1\% of cancers and while rare, it is considered terminal. Approximately 90\% of newly diagnosed MM patients will respond to their first treatment, however, the disease will eventually return and present a more resistant malignant phenotype than previously \cite{Ludwig2012a, Richardson2007}. There have been many new therapy options made available to MM patients and as a result survival has increased significantly over the past two decades. Unfortunately, as the disease progresses patients will become refractory to treatment have a median expected survival of several months. 

\subsection{MM microenvironment}
MM is an incredibly complex disease and attributing it to an overproliferation of abnormal plasma cells is a dramatic oversimplification. MM can be more accurately described as a cancer of the bone marrow with the overprofileration of abnormal plasma cells as a symptom. The hallmarks or the disease; drug resistance, immune evasion, bone lesions, even increased proliferation of abnormal plasma cells can be attributed to interactions that occur in the MM bone marrow microenvironment \cite{Damiano2000, Raab2009, Tai2006, Groen2011}. There is evidence that the transition from the premalignant, monoclonal gammopathy of uncertain significance and MM is mediated by the bone marrow microenvironment \cite{Balakumaran2010, Ribatti2014}. A better understanding of the bone marrow microenvironment and how it relates to disease progression and response to therapy is necessary for not only developing and validating treatments, but understanding why, when, and how individual patients will respond to treatment.

There is a significant body of research that focuses on interactions between MM cells and bone marrow stromal cells (BMSCs) within the context of the MM microenvironment. Much of the interactions between these two cell types have been shown to advance the malignancy of the disease. \Invitro\ studies have show that soluble factor signaling and direct cell-cell contact between MM and BMSCs lead to increased proliferation for both cell types as well as enhanced resistance of MM cells to various therapies \cite{Chauhan1996, Hideshima2002, Abdi2013}. BMSCs are also capable of supporting MM progression indirectly and has been show to play a roll in immunosuppresion, promote angiogenesis, and promote osteolysis \cite{Tai2006, Vallet2010, Ribatti2006}. BMSCs are only a single player in the MM microenvironment and form part of a complex network of factors including macrophages, mesenchymal stem cells, osteoblasts, osteoclasts, and a number of extracellular matrix components like hyaluronic acid, syndecan-1 that all seem to connect in exactly the wrong way to result in this malignant bone marrow phenotype that defines multiple myeloma \cite{Azab2009, Chatterjee2002a, Vincent2005, Lin2007, Lamorte2012, Reagan2012, Maxwell2005}. 

\subsection{Engineering for the MM Microenvironment}
Choosing which aspects of the MM microenvironment to include in an \invitro\ model from the complex web of interactions described in the literature can be challenging even when goals of the model are well-defined. Though there are some limitations which can better define a model. One such limitation is access to primary MM samples. MM is a relatively rare disease, resulting in limited access to samples. Additionally, CD138+ (a marker for MM cells) cells retrieved from bone marrow aspirates is fairly low, especially when sourced from patients currently responding to therapy where yields can be less than 100,000 cells per aspirate. Finally, primary CD138+ cells rapidly lose viability in the absence of the supportive factors of the MM microenvironment \cite{Ferrarini2013a, Zhang2014}. 

Young \textit{et al.} describes a 2D microchannel-based approach to study interactions between MM and bone marrow cells that is also amenable to deal with limited sample sizes obtained from patients \cite{Young2012}. The platform was further used as an \textit{ex vivo} model to predict patient response to bortezomib \cite{Pak2015}. At the more complex end of model spectrum, Reagan, \textit{et al.} use a tissue-engineered bone marrow microenvironment model to investigate microRNA associated with osteolysis by comparing healthy and diseased microenvironments. These two models though vastly different were able to be used to study specific phenomena in the MM microenvironment. 
