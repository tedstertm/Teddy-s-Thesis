The tissue microenvironment is a complex and dynamic system.
When microenvironments become dysregulated it can lead to a number of
diseases including cancer. Understanding the interactions in the microenvironment
is key to understanding disease progression and could lead to the development of
new and effective therapies. The complexity of the microenvironment makes the
development of \invitro\ models challenging. A model needs to have the appropriate
context to make it a good biological proxy for the system it’s designed to investigate,
but also needs to simple enough so that researchers can manipulate it to get good
data from the model.

Microfluidic techniques and technologies are amenable for complex tissue engineering tools to achieve high biological relevance and allow for precise control and manipulation over many signals experienced in the model making them well-suited for developing \invitro\ models. In spite of their benefits, microfluidics has not yet achieved widespread adoption in the biology community. To achieve higher penetration of microfluidics-based \invitro\ models on the biology bench, engineers need to design systems with not only the biology in mind, but also the biologist. This document describes how choice of material, fabrication method, and how a user interfaces with a microfluidic platform can not only lead a more biology-friendly \invitro\ model, but can enable models that would be difficult of impossible to achieve without these design considerations.