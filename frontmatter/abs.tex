At the core of engineering is balance. The balance of energy, mass, cost, ease-of-use, flexibility, precision, throughput, robustness, etc. When developing tools to provide new insights, one has to weigh the potential benefit of that new insight with the cost and expertise required to adopt or use that tool. When the various `costs' outweigh the potential gains, the tool becomes impractical. In engineering research, we often push the limits of what can be done first, before attempting to make the solution practical. This method has its place; however, a fundamental goal of the research presented here is to keep practicality and accessibility at the forefront of design, incorporating passive elements where appropriate to simplify assay designs while embracing and interfacing with existing tools when needed in order to provide practical solutions to important challenges.

The focus of this work is in the area of micro-scale cell culture assays. Thus, the underlying principles of micro-scale phenomena such as surface tension, laminar flow, reduced volumes, and evaporation are examined first to guide component design. Multiple components that leverage these phenomena are then developed to enable practical, broadly applicable, and simple functionalities including concentration of cell suspensions, laminar flow patterning, secretion detection, and oscillatory flow. Some of these components are then used to facilitate the study of circulating tumor cells, characterize tumor-cell adhesion to endothelia, and develop methods to improve delivery of stem-cell-based therapies. The modular nature of these tools is then discussed along with their potential to enable new areas of study in which multiple functionalities are required to understand complex multi-step processes such as cancer progression.